% Options for packages loaded elsewhere
\PassOptionsToPackage{unicode}{hyperref}
\PassOptionsToPackage{hyphens}{url}
\PassOptionsToPackage{dvipsnames,svgnames,x11names}{xcolor}
%
\documentclass[
]{article}

\usepackage{amsmath,amssymb}
\usepackage{lmodern}
\usepackage{iftex}
\ifPDFTeX
  \usepackage[T1]{fontenc}
  \usepackage[utf8]{inputenc}
  \usepackage{textcomp} % provide euro and other symbols
\else % if luatex or xetex
  \usepackage{unicode-math}
  \defaultfontfeatures{Scale=MatchLowercase}
  \defaultfontfeatures[\rmfamily]{Ligatures=TeX,Scale=1}
\fi
% Use upquote if available, for straight quotes in verbatim environments
\IfFileExists{upquote.sty}{\usepackage{upquote}}{}
\IfFileExists{microtype.sty}{% use microtype if available
  \usepackage[]{microtype}
  \UseMicrotypeSet[protrusion]{basicmath} % disable protrusion for tt fonts
}{}
\makeatletter
\@ifundefined{KOMAClassName}{% if non-KOMA class
  \IfFileExists{parskip.sty}{%
    \usepackage{parskip}
  }{% else
    \setlength{\parindent}{0pt}
    \setlength{\parskip}{6pt plus 2pt minus 1pt}}
}{% if KOMA class
  \KOMAoptions{parskip=half}}
\makeatother
\usepackage{xcolor}
\usepackage[margin=1.2in]{geometry}
\setlength{\emergencystretch}{3em} % prevent overfull lines
\setcounter{secnumdepth}{-\maxdimen} % remove section numbering


\providecommand{\tightlist}{%
  \setlength{\itemsep}{0pt}\setlength{\parskip}{0pt}}\usepackage{longtable,booktabs,array}
\usepackage{calc} % for calculating minipage widths
% Correct order of tables after \paragraph or \subparagraph
\usepackage{etoolbox}
\makeatletter
\patchcmd\longtable{\par}{\if@noskipsec\mbox{}\fi\par}{}{}
\makeatother
% Allow footnotes in longtable head/foot
\IfFileExists{footnotehyper.sty}{\usepackage{footnotehyper}}{\usepackage{footnote}}
\makesavenoteenv{longtable}
\usepackage{graphicx}
\makeatletter
\def\maxwidth{\ifdim\Gin@nat@width>\linewidth\linewidth\else\Gin@nat@width\fi}
\def\maxheight{\ifdim\Gin@nat@height>\textheight\textheight\else\Gin@nat@height\fi}
\makeatother
% Scale images if necessary, so that they will not overflow the page
% margins by default, and it is still possible to overwrite the defaults
% using explicit options in \includegraphics[width, height, ...]{}
\setkeys{Gin}{width=\maxwidth,height=\maxheight,keepaspectratio}
% Set default figure placement to htbp
\makeatletter
\def\fps@figure{htbp}
\makeatother

\makeatletter
\makeatother
\makeatletter
\makeatother
\makeatletter
\@ifpackageloaded{caption}{}{\usepackage{caption}}
\AtBeginDocument{%
\ifdefined\contentsname
  \renewcommand*\contentsname{Table of contents}
\else
  \newcommand\contentsname{Table of contents}
\fi
\ifdefined\listfigurename
  \renewcommand*\listfigurename{List of Figures}
\else
  \newcommand\listfigurename{List of Figures}
\fi
\ifdefined\listtablename
  \renewcommand*\listtablename{List of Tables}
\else
  \newcommand\listtablename{List of Tables}
\fi
\ifdefined\figurename
  \renewcommand*\figurename{Figure}
\else
  \newcommand\figurename{Figure}
\fi
\ifdefined\tablename
  \renewcommand*\tablename{Table}
\else
  \newcommand\tablename{Table}
\fi
}
\@ifpackageloaded{float}{}{\usepackage{float}}
\floatstyle{ruled}
\@ifundefined{c@chapter}{\newfloat{codelisting}{h}{lop}}{\newfloat{codelisting}{h}{lop}[chapter]}
\floatname{codelisting}{Listing}
\newcommand*\listoflistings{\listof{codelisting}{List of Listings}}
\makeatother
\makeatletter
\@ifpackageloaded{caption}{}{\usepackage{caption}}
\@ifpackageloaded{subcaption}{}{\usepackage{subcaption}}
\makeatother
\makeatletter
\@ifpackageloaded{tcolorbox}{}{\usepackage[many]{tcolorbox}}
\makeatother
\makeatletter
\@ifundefined{shadecolor}{\definecolor{shadecolor}{rgb}{.97, .97, .97}}
\makeatother
\makeatletter
\makeatother
\ifLuaTeX
  \usepackage{selnolig}  % disable illegal ligatures
\fi
\IfFileExists{bookmark.sty}{\usepackage{bookmark}}{\usepackage{hyperref}}
\IfFileExists{xurl.sty}{\usepackage{xurl}}{} % add URL line breaks if available
\urlstyle{same} % disable monospaced font for URLs
\hypersetup{
  pdftitle={Infinite-Valued Logic},
  pdfauthor={Phil 296 - December 5},
  colorlinks=true,
  linkcolor={blue},
  filecolor={Maroon},
  citecolor={Blue},
  urlcolor={Blue},
  pdfcreator={LaTeX via pandoc}}

\title{Infinite-Valued Logic}
\author{Phil 296 - December 5}
\date{}

\begin{document}
\maketitle
\ifdefined\Shaded\renewenvironment{Shaded}{\begin{tcolorbox}[frame hidden, enhanced, interior hidden, boxrule=0pt, breakable, borderline west={3pt}{0pt}{shadecolor}, sharp corners]}{\end{tcolorbox}}\fi

\hypertarget{three-valued-logic-without-brute-force}{%
\section{Three-Valued Logic without Brute
Force}\label{three-valued-logic-without-brute-force}}

\hypertarget{some-heuristics-for-strong-kleene}{%
\subsection{Some Heuristics for Strong
Kleene}\label{some-heuristics-for-strong-kleene}}

Assume we're working out whether \(A \vdash B\) is valid, and we don't
want to build giant truth tables. What can we do that's more efficient.

\begin{itemize}
\tightlist
\item
  There are no logical truths. So if we're just looking at whether
  \(\vdash B\), we know immediately the answer is that this doesn't
  hold.
\item
  Otherwise, we're looking for a case where the premise has value 1, and
  the conclusion does not.
\item
  If the premise is a conjunction, both conjuncts must have value 1.
\item
  If the conclusion is a disjunction, neither disjunct can have value 1,
  so both must have value less than 1.
\item
  If a premise includes just one sentence letter, that sentence can only
  have value 0 or 1. (Why?)
\item
  If the chapter 1, classical, tableau for the argument is open, it is
  not valid in Strong Kleene. (Why?)
\item
  If something is a classical logical truth, then it takes value 1 in
  Strong Kleene unless some sentence letter in it takes value 0.5. So if
  the conclusion of the argument is a classical logical truth (like
  \(p \vee \neg p\), or \(p \rightarrow p\), or
  \(\neg(p \wedge \neg p)\)), you can focus just on lines where at least
  one sentence letter in the conclusion takes value 0.5.
\item
  If something is a classical impossibility, i.e., the negation of a
  logical truth, then the maximal value it can take is 0.5.
\end{itemize}

The DeMorgan laws all hold in Strong Kleene. (These can mostly be seen
by thinking through what the connectives mean in terms of Excel
formulae.)

\begin{itemize}
\tightlist
\item
  \(\neg (A \wedge B) \vdash \neg A \vee \neg B\)
\item
  \(\neg A \vee \neg B \vdash \neg (A \wedge B)\)
\item
  \(\neg (A \vee B) \vdash \neg A \wedge \neg B\)
\item
  \(\neg A \wedge \neg B \vdash \neg (A \vee B)\)
\end{itemize}

The distribution laws all hold in Strong Kleene.

\begin{itemize}
\tightlist
\item
  \(A \wedge (B \vee C) \vdash (A \wedge B) \vee (A \wedge C)\)
\item
  \((A \wedge B) \vee (A \wedge C) \vdash A \wedge (B \vee C)\)
\item
  \(A \vee (B \wedge C) \vdash (A \vee B) \wedge (A \vee C)\)
\item
  \((A \vee B) \wedge (A \vee C) \vdash A \vee (B \vee C)\)
\end{itemize}

These of course all hold in classical logic as well, and are often
useful to remember.

And one last one to remember that's distinctive to Strong Kleene.

\begin{itemize}
\tightlist
\item
  Conditionals don't behave particularly intuitively in this logic, so
  it's usually simplest to just replace \(A \rightarrow B\) with
  \(\neg A \vee B\), and delete the double negative if \(A\) was already
  negated.
\end{itemize}

Test how well you're following these heuristics by working out which of
the following are valid in Strong Kleene.

\begin{enumerate}
\def\labelenumi{\arabic{enumi}.}
\tightlist
\item
  \(\vdash p \rightarrow p\)
\item
  \(p \vee \neg p \vdash p \rightarrow p\)
\item
  \(q \vee \neg q \vdash p \rightarrow p\)
\item
  \((p \rightarrow q) \rightarrow p \vdash p\)
\end{enumerate}

\hypertarget{some-heuristics-for-ux142ukasiewicz}{%
\subsection{Some Heuristics for
Łukasiewicz}\label{some-heuristics-for-ux142ukasiewicz}}

Because it is just like Strong Kleene when conditionals are not
involved, everything I said about DeMorgan and distribution still holds
here. And most of the bullet points at the top about Strong Kleene are
still true. But there are two that are different.

\begin{itemize}
\tightlist
\item
  There are logical truths in Łukasiewicz, such as \(p \rightarrow p\).
\item
  Some classical impossibilities can have value 1, such as
  \((p \rightarrow \neg p) \wedge (\neg p \rightarrow p)\).
\end{itemize}

Otherwise the big heuristic is one we mentioned last time.

\begin{itemize}
\tightlist
\item
  \(V(A \rightarrow B) < 1\) if and only if \(V(A) > V(B)\).
\end{itemize}

So if there is conditional in the conclusion, or a conditional is one of
the disjuncts in the conclusion, you can use this fact to put
constraints on what a `bad' line will be. Consider, for example, the
following argument.

\begin{itemize}
\tightlist
\item
  \(p \rightarrow (q \wedge r) \vdash p \rightarrow q\)
\end{itemize}

This sounds like it should be right, but it's hard to rely on intuition
in these three-value cases. We could do the 27 line table for it, but
that would be work. Here's a better argument that it works ih this
logic.

Assume that there is a valuation function, a line on the table, where
the premise is 1, and the conclusion is not 1. That is
\(V(p \rightarrow q) < 1\), so \(V(p) > V(q)\). From the definition of
\(\wedge\), we know that \(V(q) \geq V(q \wedge r)\). Putting these
inequalities together, we get \(V(p) > V(q \wedge r)\), contradicting
the assumption that \(V(p \rightarrow (q \wedge r)) = 1\).

\hypertarget{fuzzy-logic}{%
\section{Fuzzy Logic}\label{fuzzy-logic}}

\hypertarget{problems-for-three-valued-logic}{%
\subsection{Problems for Three-Valued
Logic}\label{problems-for-three-valued-logic}}

There are a few reasons to be dissatisfied with the three-valued logics
that we've looked at so far.

\begin{itemize}
\tightlist
\item
  The Strong Kleene logic has no logical truths, not even
  \(p \rightarrow p\).
\item
  The Weak Kleene logic doesn't even validate \(p \vdash p \vee q\).
\item
  The Łukasiewicz logic has strange logical truths like
  \((p \rightarrow q) \vee ((q \rightarrow r) \vee (r \rightarrow s))\).
\item
  Especially in cases of vagueness, these logics seem to treat unlike
  cases alike. Among the borderline cases of being rich, some of them
  are clearly not clearly not rich (i.e., they are a long way from the
  lower border), and some of them are clearly not clearly rich (i.e.,
  they are a long way from the upper border). But the three-valued
  logics treat all these cases alike.
\item
  In the case of vagueness, none of these theories respect so-called
  \textbf{penumbral connections}.
\end{itemize}

Here's what I mean by that last thing. Assume Alice and Bob are in the
same cultural milieu, have the same jobs, live in the same neighbourhood
etc. And within that cultural setting, they are both borderline cases of
being rich. But Alice has a few more dollars in the bank, lives in an
ever-so-slightly nicer house, makes a few hundred dollars a year more,
and generally is richer than Bob. Consider these sentences.

\begin{enumerate}
\def\labelenumi{\arabic{enumi}.}
\tightlist
\item
  If Bob is rich, so is Alice.
\item
  If Alice is rich, so is Bob.
\end{enumerate}

It seems like 1 is clearly true. If Alice is richer than Bob, which she
is, then if Bob is rich, Alice is rich too. But the converse isn't quite
as clear. Maybe that small difference between them makes a difference.
But As long as \emph{Alice is rich} and \emph{Bob is rich} both get
value 0.5, we can't tell the difference between these.

Penumbral connections are also important in law. Remember the case we
discussed in class of the vague law \emph{No vehicles in the park}. It's
a bit tricky to say what a vehicle is. But I am very confident that the
following would be bad.

\begin{itemize}
\tightlist
\item
  In the morning, someone comes before the court with a ticket for
  riding a (regular) skateboard. The magistrate upholds the ticket,
  because a (regular) skateboard is a vehicle.
\item
  In the afternoon, someone comes before the court with a ticket for
  riding a motorised skateboard. The magistrate dismisses the ticket,
  because a motorised skateboard is not a vehicle.
\end{itemize}

Whatever theory we have of how to resolve vague laws, that should be out
of bounds. So would an approach that resolved all ambiguities in the
city's favor for Black defendents, but in the defendent's favor for
white defendents. And that could be bad even if no individual case was
clearly decided wrongly. Three-valued approaches have a hard job saying
what's wrong with these pairs of decisions.

\hypertarget{more-truth-values}{%
\subsection{More Truth-Values}\label{more-truth-values}}

There is a natural solution to the last three problems: more
truth-values. So-called fuzzy logics use the following setup.

\begin{itemize}
\tightlist
\item
  The truth values are all the reals in {[}0, 1{]}. Sometimes (as in
  \emph{What If?}) you'll see people say the truth values are the
  rationals in {[}0, 1{]}, but this doesn't make a major difference, and
  I think it's cleaner to simply use the reals.
\item
  The rules for the connectives are the same as in the three-valued
  Łukasiewicz logic. So we have:
\end{itemize}

\begin{align*}
v(\neg A) &= 1 - v(A) \\
v(A \wedge B) &= min(v(A), v(B)) \\
v(A \vee B) &= max(v(A), v(B)) \\
v(A \rightarrow B) &= min(1, 1 - v(A) + v(B)) = min(1, 1 - (v(A) - v(B)))
\end{align*}

We'll get to the designated values in a second, but note that we've
already said enough to solve the last two problems above. Given what
I've said, perhaps \emph{Alice is rich} gets truth value 0.7. And
\emph{Bob is rich} gets truth value 0.4. Then we'll respect the
difference between them; the truth values represent the fact that Alice
is richer than Bob. And we get the asymmetry in the conditionals as
well. The first conditional, which is \(0.4 \rightarrow 0.7\) has value
1, while the second conditional, which is \(0.7 \rightarrow 0.4\) gets
value 0.7. That seems all to the good. But we haven't got a logic yet,
because we haven't said what the designated values are.

\hypertarget{designated-values}{%
\subsection{Designated Values}\label{designated-values}}

So, what are the designated values? Well, the phrase `fuzzy logic' is
annoyingly imprecise here, and different people use it with different
meanings on just this point.\footnote{The phrase is from a computer
  scientist Lofti Zadeh (1921-2017). But he had a somewhat idiosyncratic
  definition of what it was, which doesn't precisely correspond to
  either model here. Zadeh wasn't employed in a philosophy department,
  he was in electrical engineering at Cal, but his original paper
  setting out fuzzy logic was published in a philosophy journal,
  \emph{Synthese}. It is by some measures one of the most widely cited
  philosophy papers ever.}

One logic, what I'll call \emph{truth-preserving fuzzy logic}, simply
sets the designated value to be 1. The result is a logic that is very
similar to the three-valued Łukasiewicz logic, with the main exception
being that it no longer validates weird things like
\((p \rightarrow q) \vee ((q \rightarrow r) \vee (r \rightarrow s))\).

But a more distinctive fuzzy logic comes from taking a broader approach.
Call \emph{value-preserving fuzzy logic} the logic that has the values
and rules as above, and the following rule:

\begin{itemize}
\tightlist
\item
  An argument \(A \vdash B\) is valid iff on assignment,
  \(v(A) \leq v(B)\). That is, going from \(A\) to \(B\) never means
  going from more-true to less-true.
\end{itemize}

I think of this as a `no backwards steps' theory of validity.

\hypertarget{no-backwards-steps}{%
\subsection{No Backwards Steps}\label{no-backwards-steps}}

Note several features of this notion of validity.

\begin{enumerate}
\def\labelenumi{\arabic{enumi}.}
\tightlist
\item
  Entailment is transitive: if \(A \vdash B\) and \(B \vdash C\), then
  \(A \vdash C\).
\item
  It doesn't matter whether premises are listed or conjoined:
  \(A, B \vdash C\) iff \(A \wedge B \vdash C\).
\item
  If two sentences \(A\) and \(B\) entail each other, i.e.,
  \(A \vdash B\) and \(B \vdash A\), then they have the same truth
  values in every assignment.
\item
  As a consequence of that last point, if \(A \vdash B\) and
  \(B \vdash A\), then \(A\) and \(B\) can be substituted for each other
  in any argument. That is, let \(C, C^\prime\) be any sentences that
  are alike except that \(C\) has \(A\) as a constituent, and
  \(C^\prime\) has \(B\) in the same place. Then \(C \vdash D\) iff
  \(C^\prime \vdash D\), and \(D \vdash C\) iff \(D \vdash C^\prime\).
\end{enumerate}

Those last two properties are very important in applications of logics.
It is very helpful to be able to treat logically equivalent sentences as
if they really said the same thing in every context, and to be able to
express logical equivalence as mutual entailment.

But the three-valued logics we looked at didn't have this property.
Let's start with the three-valued Łukasiewicz logic. First question, are
these claims both true?

\begin{itemize}
\tightlist
\item
  \(p \wedge \neg p \vdash \neg (q \rightarrow q)\)
\item
  \(\neg (q \rightarrow q) \vdash p \wedge \neg p\)
\end{itemize}

Yes. In both cases the premise can't take value 1, so there is no
assignment where the premise takes value 1 and the conclusion does not.
But now look what happens when we substitute one of these sentences for
the other. Which of these are logical truths?

\begin{itemize}
\tightlist
\item
  \(\neg (q \rightarrow q) \rightarrow r\)
\item
  \((p \wedge \neg p) \rightarrow r\)
\end{itemize}

The first is a logical truth, since the antecedent always takes value 0,
and the value of a conditional with antecedent value 0 is always 1. But
the second is not a logical truth; it fails when \(p\) is 0.5 and \(r\)
is 0.

And everything I've said here applies to the infinite-valued Łukasiewicz
logic (i.e., truth-preserving fuzzy logic). I won't go over that, but
you can go back and see that none of the steps use the fact that 0, 0.5
and 1 are the only truth-values; just that they are possible
truth-values.

In Strong Kleene that example won't work since neither of them are
logical truths; rememeber there are no logical truths in Strong Kleene.
But we can still generate a problem. First, any two simple
contradictions are equivalent in Strong Kleene.

\begin{itemize}
\tightlist
\item
  \(p \wedge \neg p \vdash q \wedge \neg q\)
\item
  \(q \wedge \neg q \vdash p \wedge \neg p\)
\end{itemize}

But which of these are valid arguments in Strong Kleene?

\begin{itemize}
\tightlist
\item
  \(\neg(p \wedge \neg p) \vdash p \rightarrow p\)
\item
  \(\neg(q \wedge \neg q) \vdash p \rightarrow p\)
\end{itemize}

The first is valid. The premise is designated if \(p\) is either 0 or 1,
and in either case the conclusion is designated. But the second fails
when \(q\) is 0 or 1, and \(p\) is 0.5.

\hypertarget{modus-ponens}{%
\subsection{Modus Ponens}\label{modus-ponens}}

The substitution principle I was just describing is sometimes called
\emph{congruence}. And it's sometimes taken to be definitive of
something even being a logic. (So the three-valued logics we looked at
wouldn't even deserve the honorific \emph{logic} on this rather strict
approach.) But there is one other thing that looks just as worthy of the
name logic; that it validates this reasoning.

\begin{itemize}
\tightlist
\item
  \(A, A \rightarrow B \vdash B\)
\end{itemize}

Consider a simple instance of this where
\(A = p, B = q, V(p) = 0.9, V(q) = 0.7\). What is
\(V(p \rightarrow q)\)? Well, to get from \(p\) to \(q\) we go backwards
by 0.2, going back from 0.9 to 0.7. So
\(V(p \rightarrow q) = 1 - 0.2 = 0.8\). So the premises have values 0.9
and 0.8. But the conclusion has value 0.7. So the conclusion is less
true than either premise. So the argument is invalid.

Isn't this terrible? Isn't this the most obvious validity there is?
Well, proponents of fuzzy logic sometimes say that in fact this is a
\emph{good} feature of their logic. Why? Well, remember that this was
developed to deal with vagueness. And the canonical vagueness argument
is the Sorites argument, which looks like this.

\begin{enumerate}
\def\labelenumi{\arabic{enumi}.}
\setcounter{enumi}{-1}
\tightlist
\item
  The person with \$0 is not rich.
\item
  If the person with \$0 is not rich, neither is the person with \$1.
\item
  If the person with \$1 is not rich, neither is the person with \$2,
  etc.
\end{enumerate}

\begin{enumerate}
\def\labelenumi{\Alph{enumi}.}
\setcounter{enumi}{2}
\tightlist
\item
  The person with Bezos-wealth is not rich.
\end{enumerate}

The fuzzy logic response to this argument is that the premises are all
true, but the argument is \textbf{invalid}. We only thought it was valid
because arguments with that structure and non-vague terms are valid.

This would be a \emph{very big} change in our reasoning practices.
Imagine pulling out this move in a different class. Someone is
criticising a claim \(C\) that you make, and they say that \(A\) is
true, and if \(A\) then \(C\) is also true, so you have to give up
\(C\). And you respond that the premises there seem fine, but the
argument is not in fact valid, so \(C\) is still true. I don't know,
that seems bad. But this problem is hard!



\end{document}
