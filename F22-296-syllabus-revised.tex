% Options for packages loaded elsewhere
\PassOptionsToPackage{unicode}{hyperref}
\PassOptionsToPackage{hyphens}{url}
\PassOptionsToPackage{dvipsnames,svgnames,x11names}{xcolor}
%
\documentclass[
]{article}

\usepackage{amsmath,amssymb}
\usepackage{lmodern}
\usepackage{iftex}
\ifPDFTeX
  \usepackage[T1]{fontenc}
  \usepackage[utf8]{inputenc}
  \usepackage{textcomp} % provide euro and other symbols
\else % if luatex or xetex
  \usepackage{unicode-math}
  \defaultfontfeatures{Scale=MatchLowercase}
  \defaultfontfeatures[\rmfamily]{Ligatures=TeX,Scale=1}
  \setmainfont[BoldFont = SF Pro Rounded Semibold,Scale =
MatchLowercase]{SF Pro Thin}
  \setmathfont[]{Fira Math}
\fi
% Use upquote if available, for straight quotes in verbatim environments
\IfFileExists{upquote.sty}{\usepackage{upquote}}{}
\IfFileExists{microtype.sty}{% use microtype if available
  \usepackage[]{microtype}
  \UseMicrotypeSet[protrusion]{basicmath} % disable protrusion for tt fonts
}{}
\makeatletter
\@ifundefined{KOMAClassName}{% if non-KOMA class
  \IfFileExists{parskip.sty}{%
    \usepackage{parskip}
  }{% else
    \setlength{\parindent}{0pt}
    \setlength{\parskip}{6pt plus 2pt minus 1pt}}
}{% if KOMA class
  \KOMAoptions{parskip=half}}
\makeatother
\usepackage{xcolor}
\usepackage[margin=1.5in]{geometry}
\setlength{\emergencystretch}{3em} % prevent overfull lines
\setcounter{secnumdepth}{-\maxdimen} % remove section numbering


\providecommand{\tightlist}{%
  \setlength{\itemsep}{0pt}\setlength{\parskip}{0pt}}\usepackage{longtable,booktabs,array}
\usepackage{calc} % for calculating minipage widths
% Correct order of tables after \paragraph or \subparagraph
\usepackage{etoolbox}
\makeatletter
\patchcmd\longtable{\par}{\if@noskipsec\mbox{}\fi\par}{}{}
\makeatother
% Allow footnotes in longtable head/foot
\IfFileExists{footnotehyper.sty}{\usepackage{footnotehyper}}{\usepackage{footnote}}
\makesavenoteenv{longtable}
\usepackage{graphicx}
\makeatletter
\def\maxwidth{\ifdim\Gin@nat@width>\linewidth\linewidth\else\Gin@nat@width\fi}
\def\maxheight{\ifdim\Gin@nat@height>\textheight\textheight\else\Gin@nat@height\fi}
\makeatother
% Scale images if necessary, so that they will not overflow the page
% margins by default, and it is still possible to overwrite the defaults
% using explicit options in \includegraphics[width, height, ...]{}
\setkeys{Gin}{width=\maxwidth,height=\maxheight,keepaspectratio}
% Set default figure placement to htbp
\makeatletter
\def\fps@figure{htbp}
\makeatother

\makeatletter
\makeatother
\makeatletter
\makeatother
\makeatletter
\@ifpackageloaded{caption}{}{\usepackage{caption}}
\AtBeginDocument{%
\ifdefined\contentsname
  \renewcommand*\contentsname{Table of contents}
\else
  \newcommand\contentsname{Table of contents}
\fi
\ifdefined\listfigurename
  \renewcommand*\listfigurename{List of Figures}
\else
  \newcommand\listfigurename{List of Figures}
\fi
\ifdefined\listtablename
  \renewcommand*\listtablename{List of Tables}
\else
  \newcommand\listtablename{List of Tables}
\fi
\ifdefined\figurename
  \renewcommand*\figurename{Figure}
\else
  \newcommand\figurename{Figure}
\fi
\ifdefined\tablename
  \renewcommand*\tablename{Table}
\else
  \newcommand\tablename{Table}
\fi
}
\@ifpackageloaded{float}{}{\usepackage{float}}
\floatstyle{ruled}
\@ifundefined{c@chapter}{\newfloat{codelisting}{h}{lop}}{\newfloat{codelisting}{h}{lop}[chapter]}
\floatname{codelisting}{Listing}
\newcommand*\listoflistings{\listof{codelisting}{List of Listings}}
\makeatother
\makeatletter
\@ifpackageloaded{caption}{}{\usepackage{caption}}
\@ifpackageloaded{subcaption}{}{\usepackage{subcaption}}
\makeatother
\makeatletter
\@ifpackageloaded{tcolorbox}{}{\usepackage[many]{tcolorbox}}
\makeatother
\makeatletter
\@ifundefined{shadecolor}{\definecolor{shadecolor}{rgb}{.97, .97, .97}}
\makeatother
\makeatletter
\makeatother
\ifLuaTeX
  \usepackage{selnolig}  % disable illegal ligatures
\fi
\IfFileExists{bookmark.sty}{\usepackage{bookmark}}{\usepackage{hyperref}}
\IfFileExists{xurl.sty}{\usepackage{xurl}}{} % add URL line breaks if available
\urlstyle{same} % disable monospaced font for URLs
\hypersetup{
  pdftitle={PHIL 296: Honors Logic (Revised Timetable)},
  pdfauthor={Brian Weatherson},
  colorlinks=true,
  linkcolor={blue},
  filecolor={Maroon},
  citecolor={Blue},
  urlcolor={Blue},
  pdfcreator={LaTeX via pandoc}}

\title{PHIL 296: Honors Logic (Revised Timetable)}
\author{Brian Weatherson}
\date{Fall 2022}

\begin{document}
\maketitle
\ifdefined\Shaded\renewenvironment{Shaded}{\begin{tcolorbox}[boxrule=0pt, sharp corners, breakable, frame hidden, interior hidden, borderline west={3pt}{0pt}{shadecolor}, enhanced]}{\end{tcolorbox}}\fi

\textbf{Lead Instructor}: Brian Weatherson\\
\includegraphics[width=1em,height=1em]{figs/fa-icon-851abd0b78479f510d18caf2daa1d792.pdf}
weath@umich.edu\\
\includegraphics[width=1.25em,height=1em]{figs/fa-icon-4afc4ec86e4719f7fac10dc1a986382a.pdf}
canvas.umich.edu\\
\strut \\
\textbf{Office Hours}: TBA\\

\hypertarget{course-description}{%
\section{Course Description}\label{course-description}}

This course is a fairly fast-paced introduction to many different
logics. We will start with a very quick discussion of \textbf{classical
logic}, the logic that is used in most other courses (both in philosophy
and elsewhere). We'll then move on to \textbf{modal logic}, the logic of
possibility and necessity. And finally, we'll look at rivals to
classical logic, the imaginatively named \textbf{non-classical logics}.
The aim of the course is to both give you a working familiarity with the
technical aspects of these logics, but also some background about why
one might care about rivals to classical logic, and when one might think
these rivals are suitable to use.

\hypertarget{canvas}{%
\section{Canvas}\label{canvas}}

There is a Canvas site for this course, which can be accessed from
\url{https://canvas.umich.edu}. Course documents (syllabus, lecture
notes, assignments) will be available from this site. Please make sure
that you can access this site. Consult the site regularly for
announcements, including changes to the course schedule. And there are
many tools on the site to communicate with each other, and with me.

\newpage

\hypertarget{required-materials}{%
\section{Required Materials}\label{required-materials}}

The bulk of the course will be spent working through selections from

\begin{itemize}
\tightlist
\item
  \emph{An Introduction to Non-Classical Logic: From If to Is}, by
  Graham Priest Cambridge University Press, 2008. Second Edition.
\end{itemize}

It is not important that you have the physical version of the book; the
electronic version is fine. But it is important that you have the second
edition; the first edition only has half the material.

\hypertarget{course-requirements}{%
\section{Course Requirements}\label{course-requirements}}

\begin{enumerate}
\def\labelenumi{\arabic{enumi}.}
\tightlist
\item
  Participate in class. We will spend a bit of time in class going over
  problems, and it is important to practice these.
\item
  Do weekly online quizzes. These will be available through Canvas.
\item
  Do a final exam. This will also be through Canvas.
\end{enumerate}

\hypertarget{grade-breakdown}{%
\section{Grade Breakdown}\label{grade-breakdown}}

\begin{itemize}
\tightlist
\item
  Class participation: 10\%;
\item
  Ten Quizzes, of which the best 8 will count: 7.5\% each, for a total
  of 60\%;
\item
  Final Exam: 30\%.
\end{itemize}

The quizzes will be due on Friday, and will mostly cover the material
from that week. (The exceptions are noted in the text below.) The exam
will be held through Canvas, and will be due on the scheduled date of
the exam. The schedule is at
\url{https://ro.umich.edu/calendars/final-exams/2022-fall}, and I
believe that means we're to be done by 6pm on Monday, December 12.

\newpage

This is the revised timetable for the second half of the term; after the
first one was considerably too ambitious. I hope we can stick to this
one, though it again we'll see if we have to adjust

\hypertarget{wednesday-october-19}{%
\subsection{Wednesday, October 19}\label{wednesday-october-19}}

\begin{description}
\tightlist
\item[Topic]
Reading a model off an open tableau in modal logic
\item[Required Reading]
Priest, §§2.4.7-2.4.9
\item[Weekly Quiz]
Due 5pm, Friday, October 21
\end{description}

\hypertarget{monday-october-24}{%
\subsection{Monday, October 24}\label{monday-october-24}}

\begin{description}
\tightlist
\item[Topic]
Three more modal logics (plus combinations)
\item[Required Reading]
Priest, §3.2
\end{description}

\hypertarget{wednesday-october-26}{%
\subsection{Wednesday, October 26}\label{wednesday-october-26}}

\begin{description}
\tightlist
\item[Topic]
Tableau for these new logics
\item[Required Reading]
Priest, §3.3-4
\item[Weekly Quiz]
Due 5pm, Friday, October 28
\end{description}

\hypertarget{monday-october-31}{%
\subsection{Monday, October 31}\label{monday-october-31}}

\begin{description}
\tightlist
\item[Topic]
The special logic S5
\item[Required Reading]
Priest, §3.5
\end{description}

\hypertarget{wednesday-november-02}{%
\subsection{Wednesday, November 02}\label{wednesday-november-02}}

\begin{description}
\tightlist
\item[Topic]
Applications
\item[Required Reading]
Not in the book!
\item[Weekly Quiz]
Due 5pm, Friday, November 04
\end{description}

\hypertarget{monday-november-07}{%
\subsection{Monday, November 07}\label{monday-november-07}}

\begin{description}
\tightlist
\item[Topic]
Strict Conditionals
\item[Required Reading]
Priest, §4.5-4.9
\end{description}

\hypertarget{wednesday-november-09}{%
\subsection{Wednesday, November 09}\label{wednesday-november-09}}

\begin{description}
\tightlist
\item[Topic]
Problems for Strict Conditionals
\item[Required Reading]
Priest, §5.1-5.2
\item[Weekly Quiz]
Due 5pm, Friday, November 11
\end{description}

\hypertarget{monday-november-14}{%
\subsection{Monday, November 14}\label{monday-november-14}}

\begin{description}
\tightlist
\item[Topic]
Conditional Semantics
\item[Required Reading]
Priest, §5.3
\end{description}

\hypertarget{wednesday-november-16}{%
\subsection{Wednesday, November 16}\label{wednesday-november-16}}

\begin{description}
\tightlist
\item[Topic]
Tableau for conditionals
\item[Required Reading]
Priest, §5.4
\item[Weekly Quiz]
Due 5pm, Friday, November 18
\end{description}

\hypertarget{monday-november-21}{%
\subsection{Monday, November 21}\label{monday-november-21}}

\begin{description}
\tightlist
\item[Topic]
Similarity Spheres
\item[Required Reading]
Priest, §5.5-5.6
\item[Weekly Quiz]
None this week
\end{description}

\hypertarget{monday-november-28}{%
\subsection{Monday, November 28}\label{monday-november-28}}

\begin{description}
\tightlist
\item[Topic]
C1 and C2
\item[Required Reading]
Priest, §5.7-5.8
\end{description}

\hypertarget{wednesday-november-30}{%
\subsection{Wednesday, November 30}\label{wednesday-november-30}}

\begin{description}
\tightlist
\item[Topic]
Finishing up chapter 5
\item[Required Reading]
Priest, Ch. 5
\item[Weekly Quiz]
Due 5pm, Friday, December 02
\end{description}

\hypertarget{monday-december-05}{%
\subsection{Monday, December 05}\label{monday-december-05}}

\begin{description}
\tightlist
\item[Topic]
Review
\item[Required Reading]
None
\end{description}

\begin{itemize}
\tightlist
\item
  The last week will be for review, and for extra time for any topic we
  felt was too rushed through the course.
\end{itemize}

\hypertarget{plagiarism}{%
\section{Plagiarism}\label{plagiarism}}

Although team-work is encouraged, plagiarism is strictly prohibited. You
are responsible for making sure that none of your work is plagiarized.
Be sure to cite work that you use, both direct quotations and
paraphrased ideas. Any citation method that is tolerably clear is
permitted, but if you'd like a good description of a citation scheme
that works well in philosophy, look at \url{http://bit.ly/VDhRJ4}.

You are encouraged to discuss the course material, including
assignments, with your classmates, but all written work that you hand in
under your own name must be your own.

You should also be familiar with the academic integrity policies of the
College of Literature, Science \& the Arts at the University of
Michigan, which are available here:
\url{http://www.lsa.umich.edu/academicintegrity/}. Violations of these
policies will be reported to the Office of the Assistant Dean for
Student Academic Affairs, and sanctioned with a course grade of F.

\hypertarget{disability}{%
\section{Disability}\label{disability}}

The University of Michigan abides by the Americans with Disabilities Act
of 1990, Section 504 of the Rehabilitation Act of 1973, and other
applicable federal and state laws that prohibit discrimination on the
basis of disability, which mandate that reasonable accommodations be
provided for qualified students with disabilities.

If you have a disability, and may require some type of instructional
and/or examination accommodation, please contact me early in the
semester. If you have not already done so, you will also need to
register with the Office of Services for Students with Disabilities. The
office is located at G664 Haven Hall.

For more information on disability services at the University of
Michigan, go to \url{http://ssd.umich.edu}.



\end{document}
