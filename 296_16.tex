% Options for packages loaded elsewhere
\PassOptionsToPackage{unicode}{hyperref}
\PassOptionsToPackage{hyphens}{url}
%
\documentclass[
  14pt,
  letterpaper,
  ignorenonframetext,
]{beamer}
\usepackage{pgfpages}
\setbeamertemplate{caption}[numbered]
\setbeamertemplate{caption label separator}{: }
\setbeamercolor{caption name}{fg=normal text.fg}
\beamertemplatenavigationsymbolsempty
% Prevent slide breaks in the middle of a paragraph
\widowpenalties 1 10000
\raggedbottom
\setbeamertemplate{part page}{
  \centering
  \begin{beamercolorbox}[sep=16pt,center]{part title}
    \usebeamerfont{part title}\insertpart\par
  \end{beamercolorbox}
}
\setbeamertemplate{section page}{
  \centering
  \begin{beamercolorbox}[sep=12pt,center]{part title}
    \usebeamerfont{section title}\insertsection\par
  \end{beamercolorbox}
}
\setbeamertemplate{subsection page}{
  \centering
  \begin{beamercolorbox}[sep=8pt,center]{part title}
    \usebeamerfont{subsection title}\insertsubsection\par
  \end{beamercolorbox}
}
\AtBeginPart{
  \frame{\partpage}
}
\AtBeginSection{
  \ifbibliography
  \else
    \frame{\sectionpage}
  \fi
}
\AtBeginSubsection{
  \frame{\subsectionpage}
}

\usepackage{amsmath,amssymb}
\usepackage{lmodern}
\usepackage{iftex}
\ifPDFTeX
  \usepackage[T1]{fontenc}
  \usepackage[utf8]{inputenc}
  \usepackage{textcomp} % provide euro and other symbols
\else % if luatex or xetex
  \usepackage{unicode-math}
  \defaultfontfeatures{Scale=MatchLowercase}
  \defaultfontfeatures[\rmfamily]{Ligatures=TeX,Scale=1}
  \setmainfont[Scale=MatchLowercase]{SF Pro Text Light}
  \setmathfont[]{STIX Two Math}
\fi
\usetheme[]{boxes}
\usecolortheme{wolverine}
\usefonttheme{serif} % use mainfont rather than sansfont for slide text
% Use upquote if available, for straight quotes in verbatim environments
\IfFileExists{upquote.sty}{\usepackage{upquote}}{}
\IfFileExists{microtype.sty}{% use microtype if available
  \usepackage[]{microtype}
  \UseMicrotypeSet[protrusion]{basicmath} % disable protrusion for tt fonts
}{}
\makeatletter
\@ifundefined{KOMAClassName}{% if non-KOMA class
  \IfFileExists{parskip.sty}{%
    \usepackage{parskip}
  }{% else
    \setlength{\parindent}{0pt}
    \setlength{\parskip}{6pt plus 2pt minus 1pt}}
}{% if KOMA class
  \KOMAoptions{parskip=half}}
\makeatother
\usepackage{xcolor}
\newif\ifbibliography
\setlength{\emergencystretch}{3em} % prevent overfull lines
\setcounter{secnumdepth}{-\maxdimen} % remove section numbering


\providecommand{\tightlist}{%
  \setlength{\itemsep}{0pt}\setlength{\parskip}{0pt}}\usepackage{longtable,booktabs,array}
\usepackage{calc} % for calculating minipage widths
\usepackage{caption}
% Make caption package work with longtable
\makeatletter
\def\fnum@table{\tablename~\thetable}
\makeatother
\usepackage{graphicx}
\makeatletter
\def\maxwidth{\ifdim\Gin@nat@width>\linewidth\linewidth\else\Gin@nat@width\fi}
\def\maxheight{\ifdim\Gin@nat@height>\textheight\textheight\else\Gin@nat@height\fi}
\makeatother
% Scale images if necessary, so that they will not overflow the page
% margins by default, and it is still possible to overwrite the defaults
% using explicit options in \includegraphics[width, height, ...]{}
\setkeys{Gin}{width=\maxwidth,height=\maxheight,keepaspectratio}
% Set default figure placement to htbp
\makeatletter
\def\fps@figure{htbp}
\makeatother

\captionsetup[figure]{labelformat=empty}
\usepackage{pgfpages}
\setbeamertemplate{itemize item}[circle]
\setbeamertemplate{footline}[frame number]{}
\mode<handout>{\pgfpagesuselayout{6 on 1}[letterpaper, border shrink=8mm]}
\AtBeginSection{%
   \begin{frame}
       \tableofcontents[currentsection]
   \end{frame}
}
\usepackage{tikz}
\usetikzlibrary{positioning,arrows,calc}
\tikzset{
  modal/.style={>=stealth',
    shorten >=1pt,
    shorten <=1pt,
    auto,
   node distance=1.5cm,
   label distance=2pt,
   semithick},
 every label/.style={phantom,align=left},
 world/.style = {circle,draw,minimum size=0.5cm,fill=gray!15},
 modal every node/.style={world},
 point/.style={circle,draw,inner sep=0.5mm,fill=black},
 phantom/.style={rectangle,inner sep=0pt,draw=none,fill=none},
 reflexive above/.style={->,loop,looseness=7,in=60,out=120},
 reflexive below/.style={->,loop,looseness=7,in=240,out=300},
 reflexive left/.style={->,loop,looseness=7,in=150,out=210},
 reflexive right/.style={->,loop,looseness=7,in=30,out=330}}
\makeatletter
\makeatother
\makeatletter
\makeatother
\makeatletter
\@ifpackageloaded{caption}{}{\usepackage{caption}}
\AtBeginDocument{%
\ifdefined\contentsname
  \renewcommand*\contentsname{Table of contents}
\else
  \newcommand\contentsname{Table of contents}
\fi
\ifdefined\listfigurename
  \renewcommand*\listfigurename{List of Figures}
\else
  \newcommand\listfigurename{List of Figures}
\fi
\ifdefined\listtablename
  \renewcommand*\listtablename{List of Tables}
\else
  \newcommand\listtablename{List of Tables}
\fi
\ifdefined\figurename
  \renewcommand*\figurename{Figure}
\else
  \newcommand\figurename{Figure}
\fi
\ifdefined\tablename
  \renewcommand*\tablename{Table}
\else
  \newcommand\tablename{Table}
\fi
}
\@ifpackageloaded{float}{}{\usepackage{float}}
\floatstyle{ruled}
\@ifundefined{c@chapter}{\newfloat{codelisting}{h}{lop}}{\newfloat{codelisting}{h}{lop}[chapter]}
\floatname{codelisting}{Listing}
\newcommand*\listoflistings{\listof{codelisting}{List of Listings}}
\makeatother
\makeatletter
\@ifpackageloaded{caption}{}{\usepackage{caption}}
\@ifpackageloaded{subcaption}{}{\usepackage{subcaption}}
\makeatother
\makeatletter
\@ifpackageloaded{tcolorbox}{}{\usepackage[many]{tcolorbox}}
\makeatother
\makeatletter
\@ifundefined{shadecolor}{\definecolor{shadecolor}{rgb}{.97, .97, .97}}
\makeatother
\makeatletter
\makeatother
\ifLuaTeX
  \usepackage{selnolig}  % disable illegal ligatures
\fi
\IfFileExists{bookmark.sty}{\usepackage{bookmark}}{\usepackage{hyperref}}
\IfFileExists{xurl.sty}{\usepackage{xurl}}{} % add URL line breaks if available
\urlstyle{same} % disable monospaced font for URLs
\hypersetup{
  pdftitle={Honors Logic, Lecture 16 - Modal Logic},
  pdfauthor={Brian Weatherson},
  hidelinks,
  pdfcreator={LaTeX via pandoc}}

\title{Honors Logic, Lecture 16 - Modal Logic}
\author{Brian Weatherson}
\date{2022-10-26}

\begin{document}
\frame{\titlepage}
\ifdefined\Shaded\renewenvironment{Shaded}{\begin{tcolorbox}[sharp corners, boxrule=0pt, breakable, enhanced, interior hidden, borderline west={3pt}{0pt}{shadecolor}, frame hidden]}{\end{tcolorbox}}\fi

\begin{frame}{Why Do Modal Logic?}
\protect\hypertarget{why-do-modal-logic}{}
\begin{itemize}
\tightlist
\item
  In particular, why not just use first order logic to quantify over
  possibilities?
\item
  Instead of saying \(\Box p\), say \(\forall w: p(w)\) or something?
\item
  This is a non-rhetorical question; in lots of situations we do say
  something more like \(\forall w: p(w)\).
\end{itemize}
\end{frame}

\begin{frame}{Why Do Modal Logic?}
\protect\hypertarget{why-do-modal-logic-1}{}
Historically, four reasons.

\begin{enumerate}[<+->]
\tightlist
\item
  Scepticism about the \(w\), such as Prior on time.
\item
  Scepticism about \(p(w)\), as in worries about truth.
\item
  More natural/intuitive to talk about \(\Box\) than about \(r\).
\item
  We want to understand \(\Box\), and thinking about \(r\) helps us do
  that.
\end{enumerate}
\end{frame}

\begin{frame}{\(\Diamond p \supset \Box \Diamond p\)}
\protect\hypertarget{diamond-p-supset-box-diamond-p}{}
We are going to do this in K, then in K\(\rho\), then in K\(\rho \tau\),
then in K\(\rho \sigma\), then finally in K\(\rho \sigma \tau\), which
is equivalent to K\(\upsilon\).
\end{frame}

\begin{frame}
Start with the tableau for K.

\begin{center}
\neg($\Diamond p \supset \Box \Diamond p$), 0 \\
$\Diamond p, 0$ \\
$\neg \Box \Diamond p, 0$ \\
$\Diamond \neg \Diamond p, 0$ \\
$0r1$ \\
$p, 1$ \\
$0r2$ \\
$\neg \Diamond p, 2$ \\
$\Box \neg p, 2$
\end{center}
\end{frame}

\begin{frame}
Now let's extend the tableau for K\(\rho\).

\begin{columns}[T]
\begin{column}{0.48\textwidth}
\begin{center}
\neg($\Diamond p \supset \Box \Diamond p$), 0 \\
$\Diamond p, 0$ \\
$\neg \Box \Diamond p, 0$ \\
$\Diamond \neg \Diamond p, 0$ \\
$0r1$ \\
$p, 1$ \\
$0r2$ \\
$\neg \Diamond p, 2$ \\
$\Box \neg p, 2$
\end{center}
\end{column}

\begin{column}{0.48\textwidth}
\begin{center}
$0r0$ \\
$1r1$ \\
$2r2$ \\
$\neg p, 2$
\end{center}
\end{column}
\end{columns}
\end{frame}

\begin{frame}
That is also a tableau for K\(\rho \tau\).

\begin{columns}[T]
\begin{column}{0.48\textwidth}
\begin{center}
\neg($\Diamond p \supset \Box \Diamond p$), 0 \\
$\Diamond p, 0$ \\
$\neg \Box \Diamond p, 0$ \\
$\Diamond \neg \Diamond p, 0$ \\
$0r1$ \\
$p, 1$ \\
$0r2$ \\
$\neg \Diamond p, 2$ \\
$\Box \neg p, 2$
\end{center}
\end{column}

\begin{column}{0.48\textwidth}
\begin{center}
$0r0$ \\
$1r1$ \\
$2r2$ \\
$\neg p, 2$
\end{center}
\end{column}
\end{columns}
\end{frame}

\begin{frame}
To do the tableau for K\(\rho \sigma\), we need a couple more lines, but
it's still open.

\begin{columns}[T]
\begin{column}{0.48\textwidth}
\begin{center}
\neg($\Diamond p \supset \Box \Diamond p$), 0 \\
$\Diamond p, 0$ \\
$\neg \Box \Diamond p, 0$ \\
$\Diamond \neg \Diamond p, 0$ \\
$0r1$ \\
$p, 1$ \\
$0r2$ \\
$\neg \Diamond p, 2$ \\
$\Box \neg p, 2$
\end{center}
\end{column}

\begin{column}{0.48\textwidth}
\begin{center}
$0r0$ \\
$1r1$ \\
$2r2$ \\
$\neg p, 2$ \\
$2r0$ \\
$1r0$ \\
$\neg p, 0$
\end{center}
\end{column}
\end{columns}
\end{frame}

\begin{frame}
To do the tableau for K\(\rho \sigma \tau\), we need more lines, and now
it closes.

\begin{columns}[T]
\begin{column}{0.48\textwidth}
\begin{center}
\neg($\Diamond p \supset \Box \Diamond p$), 0 \\
$\Diamond p, 0$ \\
$\neg \Box \Diamond p, 0$ \\
$\Diamond \neg \Diamond p, 0$ \\
$0r1$ \\
$p, 1$ \\
$0r2$ \\
$\neg \Diamond p, 2$ \\
$\Box \neg p, 2$
\end{center}
\end{column}

\begin{column}{0.48\textwidth}
\begin{center}
$0r0$ \\
$1r1$ \\
$2r2$ \\
$\neg p, 2$ \\
$2r0$ \\
$1r0$ \\
$\neg p, 0$ \\
$2r1$ \\
$\neg p, 1$ \\
$x$
\end{center}
\end{column}
\end{columns}
\end{frame}

\begin{frame}
The tableau for K\(\upsilon\), is a bit simpler, because it doesn't
include \(r\) lines.

\begin{center}
\neg($\Diamond p \supset \Box \Diamond p$), 0 \\
$\Diamond p, 0$ \\
$\neg \Box \Diamond p, 0$ \\
$\Diamond \neg \Diamond p, 0$ \\
$p, 1$ \\
$\neg \Diamond p, 2$ \\
$\Box \neg p, 2$ \\
$\neg p, 2$ \\
$\neg p, 0$ \\
$\neg p, 1$ \\
$x$
\end{center}
\end{frame}

\begin{frame}{\(\Box(\Box p \supset q) \vee \Box(\Box q \supset p)\)}
\protect\hypertarget{boxbox-p-supset-q-vee-boxbox-q-supset-p}{}
This is not very intuitive, but it is a little interesting
mathematically.

\begin{itemize}
\tightlist
\item
  It's the \textbf{characteristic axiom} of models that are `linear'.
\item
  I don't \emph{think} I'll have time today to say what a characteristic
  axiom is; the priority today is getting the formalism down.
\item
  But hopefully we'll have some time for it next week.
\end{itemize}
\end{frame}

\begin{frame}
Start with an open tableau in K.

\begin{center}
$\neg(\Box(\Box p \supset q) \vee \Box(\Box q \supset p)), 0$ \\
$\neg \Box(\Box p \supset q), 0$ \\
$\neg \Box(\Box q \supset p), 0$ \\
$\Diamond \neg(\Box p \supset q), 0$ \\
$\Diamond \neg(\Box q \supset p), 0$ \\
$\neg(\Box p \supset q), 1$ \\
$0r1$ \\
$\neg(\Box q \supset p), 2$ \\
$0r2$ \\
$\Box p, 1$ \\
$\neg q, 1$ \\
$\Box q, 2$ \\
$\neg p, 2$ \\
\end{center}
\end{frame}

\begin{frame}
Now onto K\(\rho\), which is still open.

\begin{columns}[T]
\begin{column}{0.65\textwidth}
\begin{center}
$\neg(\Box(\Box p \supset q) \vee \Box(\Box q \supset p)), 0$ \\
$\neg \Box(\Box p \supset q), 0$ \\
$\neg \Box(\Box q \supset p), 0$ \\
$\Diamond \neg(\Box p \supset q), 0$ \\
$\Diamond \neg(\Box q \supset p), 0$ \\
$\neg(\Box p \supset q), 1$ \\
$0r1$ \\
$\neg(\Box q \supset p), 2$ \\
$0r2$ \\
$\Box p, 1$ \\
$\neg q, 1$ \\
$\Box q, 2$ \\
$\neg p, 2$ \\
\end{center}
\end{column}

\begin{column}{0.35\textwidth}
\begin{center}
$0r0$ \\
$1r1$ \\
$2r2$ \\
$p, 1$ \\
$q, 2$
\end{center}
\end{column}
\end{columns}
\end{frame}

\begin{frame}
That's also an open tableau for K\(\rho \tau\).

\begin{columns}[T]
\begin{column}{0.65\textwidth}
\begin{center}
$\neg(\Box(\Box p \supset q) \vee \Box(\Box q \supset p)), 0$ \\
$\neg \Box(\Box p \supset q), 0$ \\
$\neg \Box(\Box q \supset p), 0$ \\
$\Diamond \neg(\Box p \supset q), 0$ \\
$\Diamond \neg(\Box q \supset p), 0$ \\
$\neg(\Box p \supset q), 1$ \\
$0r1$ \\
$\neg(\Box q \supset p), 2$ \\
$0r2$ \\
$\Box p, 1$ \\
$\neg q, 1$ \\
$\Box q, 2$ \\
$\neg p, 2$ \\
\end{center}
\end{column}

\begin{column}{0.35\textwidth}
\begin{center}
$0r0$ \\
$1r1$ \\
$2r2$ \\
$p, 1$ \\
$q, 2$
\end{center}
\end{column}
\end{columns}
\end{frame}

\begin{frame}
K\(\rho \sigma\) needs a few more lines, but is still open.

\begin{columns}[T]
\begin{column}{0.65\textwidth}
\begin{center}
$\neg(\Box(\Box p \supset q) \vee \Box(\Box q \supset p)), 0$ \\
$\neg \Box(\Box p \supset q), 0$ \\
$\neg \Box(\Box q \supset p), 0$ \\
$\Diamond \neg(\Box p \supset q), 0$ \\
$\Diamond \neg(\Box q \supset p), 0$ \\
$\neg(\Box p \supset q), 1$ \\
$0r1$ \\
$\neg(\Box q \supset p), 2$ \\
$0r2$ \\
$\Box p, 1$ \\
$\neg q, 1$ \\
$\Box q, 2$ \\
$\neg p, 2$ \\
\end{center}
\end{column}

\begin{column}{0.35\textwidth}
\begin{center}
$0r0$ \\
$1r1$ \\
$2r2$ \\
$p, 1$ \\
$q, 2$ \\
$1r0$ \\
$2r0$ \\
$p, 0$ \\
$q, 0$
\end{center}
\end{column}
\end{columns}
\end{frame}

\begin{frame}
But K\(\rho \sigma \tau\) closes

\begin{columns}[T]
\begin{column}{0.65\textwidth}
\begin{center}
$\neg(\Box(\Box p \supset q) \vee \Box(\Box q \supset p)), 0$ \\
$\neg \Box(\Box p \supset q), 0$ \\
$\neg \Box(\Box q \supset p), 0$ \\
$\Diamond \neg(\Box p \supset q), 0$ \\
$\Diamond \neg(\Box q \supset p), 0$ \\
$\neg(\Box p \supset q), 1$ \\
$0r1$ \\
$\neg(\Box q \supset p), 2$ \\
$0r2$ \\
$\Box p, 1$ \\
$\neg q, 1$ \\
$\Box q, 2$ \\
$\neg p, 2$ \\
\end{center}
\end{column}

\begin{column}{0.35\textwidth}
\begin{center}
$0r0$ \\
$1r1$ \\
$2r2$ \\
$p, 1$ \\
$q, 2$ \\
$1r0$ \\
$2r0$ \\
$p, 0$ \\
$q, 0$ \\
$1r2$ \\
$p, 2$ \\
$x$
\end{center}
\end{column}
\end{columns}
\end{frame}

\begin{frame}{\(\Box p \supset \Box \Box p\)}
\protect\hypertarget{box-p-supset-box-box-p}{}
Next example is this one, which is the characteristic axiom of
transitive models.
\end{frame}

\begin{frame}
Let's skip K and go straight to K\(\rho\), since it's open.

\begin{center}
$\neg(\Box p \supset \Box \Box p), 0$ \\
$\Box p, 0$ \\
$\neg \Box \Box p, 0$ \\
$0r0$ \\
$p, 0$ \\
$\Diamond \neg \Box p, 0$ \\
$0r1$ \\
$\neg \Box p, 1$ \\
$p, 1$ \\
$1r1$ \\
$\Diamond \neg p, 1$ \\
$1r2$ \\
$\neg p, 2$ \\
$2r2$
\end{center}
\end{frame}

\begin{frame}
It takes just a bit more work to get K\(\rho \sigma\).

\begin{columns}[T]
\begin{column}{0.65\textwidth}
\begin{center}
$\neg(\Box p \supset \Box \Box p), 0$ \\
$\Box p, 0$ \\
$\neg \Box \Box p, 0$ \\
$0r0$ \\
$p, 0$ \\
$\Diamond \neg \Box p, 0$ \\
$0r1$ \\
$\neg \Box p, 1$ \\
$p, 1$ \\
$1r1$ \\
\end{center}
\end{column}

\begin{column}{0.35\textwidth}
\begin{center}
$\Diamond \neg p, 1$ \\
$1r2$ \\
$\neg p, 2$ \\
$2r2$ \\
$\mathbf{2r1}$ \\
$\mathbf{1r0}$ \\
\end{center}
\end{column}
\end{columns}

I bolded the new lines, but they don't have any interesting
implications.
\end{frame}

\begin{frame}
But this closes in K\(\rho \tau\), and hence in K\(\rho \sigma \tau\)
and in K\(\upsilon\).

\begin{columns}[T]
\begin{column}{0.65\textwidth}
\begin{center}
$\neg(\Box p \supset \Box \Box p), 0$ \\
$\Box p, 0$ \\
$\neg \Box \Box p, 0$ \\
$0r0$ \\
$p, 0$ \\
$\Diamond \neg \Box p, 0$ \\
$0r1$ \\
$\neg \Box p, 1$ \\
$p, 1$ \\
$1r1$ \\
\end{center}
\end{column}

\begin{column}{0.35\textwidth}
\begin{center}
$\Diamond \neg p, 1$ \\
$1r2$ \\
$\neg p, 2$ \\
$2r2$ \\
$\mathbf{0r2}$ \\
$\mathbf{p, 2}$ \\
$x$
\end{center}
\end{column}
\end{columns}

And we get \(p\) is both true and false at 2.
\end{frame}

\begin{frame}{\(\neg \Box \Diamond p\)}
\protect\hypertarget{neg-box-diamond-p}{}
This obviously isn't a logical truth!

\begin{itemize}
\tightlist
\item
  But the tableau for it in K\(\upsilon\) is annoying.
\end{itemize}
\end{frame}

\begin{frame}
\begin{center}
$\neg \neg \Box \Diamond p, 0$ \\
$\Box \Diamond p, 0$ \\
$\Diamond p, 0$ \\
$p, 1$ \\
$\Diamond p, 1$ \\
$p, 2$ \\
$\Diamond p, 2$ \\
$\dots$
\end{center}

A model

\begin{itemize}
\tightlist
\item
  There are infinitely many points.
\item
  They can all access each other.
\item
  \(p\) is true at all of them.
\end{itemize}
\end{frame}



\end{document}
