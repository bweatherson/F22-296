% Options for packages loaded elsewhere
\PassOptionsToPackage{unicode}{hyperref}
\PassOptionsToPackage{hyphens}{url}
\PassOptionsToPackage{dvipsnames,svgnames,x11names}{xcolor}
%
\documentclass[
]{article}

\usepackage{amsmath,amssymb}
\usepackage{lmodern}
\usepackage{iftex}
\ifPDFTeX
  \usepackage[T1]{fontenc}
  \usepackage[utf8]{inputenc}
  \usepackage{textcomp} % provide euro and other symbols
\else % if luatex or xetex
  \usepackage{unicode-math}
  \defaultfontfeatures{Scale=MatchLowercase}
  \defaultfontfeatures[\rmfamily]{Ligatures=TeX,Scale=1}
\fi
% Use upquote if available, for straight quotes in verbatim environments
\IfFileExists{upquote.sty}{\usepackage{upquote}}{}
\IfFileExists{microtype.sty}{% use microtype if available
  \usepackage[]{microtype}
  \UseMicrotypeSet[protrusion]{basicmath} % disable protrusion for tt fonts
}{}
\makeatletter
\@ifundefined{KOMAClassName}{% if non-KOMA class
  \IfFileExists{parskip.sty}{%
    \usepackage{parskip}
  }{% else
    \setlength{\parindent}{0pt}
    \setlength{\parskip}{6pt plus 2pt minus 1pt}}
}{% if KOMA class
  \KOMAoptions{parskip=half}}
\makeatother
\usepackage{xcolor}
\usepackage[margin=1.2in]{geometry}
\setlength{\emergencystretch}{3em} % prevent overfull lines
\setcounter{secnumdepth}{-\maxdimen} % remove section numbering


\providecommand{\tightlist}{%
  \setlength{\itemsep}{0pt}\setlength{\parskip}{0pt}}\usepackage{longtable,booktabs,array}
\usepackage{calc} % for calculating minipage widths
% Correct order of tables after \paragraph or \subparagraph
\usepackage{etoolbox}
\makeatletter
\patchcmd\longtable{\par}{\if@noskipsec\mbox{}\fi\par}{}{}
\makeatother
% Allow footnotes in longtable head/foot
\IfFileExists{footnotehyper.sty}{\usepackage{footnotehyper}}{\usepackage{footnote}}
\makesavenoteenv{longtable}
\usepackage{graphicx}
\makeatletter
\def\maxwidth{\ifdim\Gin@nat@width>\linewidth\linewidth\else\Gin@nat@width\fi}
\def\maxheight{\ifdim\Gin@nat@height>\textheight\textheight\else\Gin@nat@height\fi}
\makeatother
% Scale images if necessary, so that they will not overflow the page
% margins by default, and it is still possible to overwrite the defaults
% using explicit options in \includegraphics[width, height, ...]{}
\setkeys{Gin}{width=\maxwidth,height=\maxheight,keepaspectratio}
% Set default figure placement to htbp
\makeatletter
\def\fps@figure{htbp}
\makeatother

\makeatletter
\makeatother
\makeatletter
\makeatother
\makeatletter
\@ifpackageloaded{caption}{}{\usepackage{caption}}
\AtBeginDocument{%
\ifdefined\contentsname
  \renewcommand*\contentsname{Table of contents}
\else
  \newcommand\contentsname{Table of contents}
\fi
\ifdefined\listfigurename
  \renewcommand*\listfigurename{List of Figures}
\else
  \newcommand\listfigurename{List of Figures}
\fi
\ifdefined\listtablename
  \renewcommand*\listtablename{List of Tables}
\else
  \newcommand\listtablename{List of Tables}
\fi
\ifdefined\figurename
  \renewcommand*\figurename{Figure}
\else
  \newcommand\figurename{Figure}
\fi
\ifdefined\tablename
  \renewcommand*\tablename{Table}
\else
  \newcommand\tablename{Table}
\fi
}
\@ifpackageloaded{float}{}{\usepackage{float}}
\floatstyle{ruled}
\@ifundefined{c@chapter}{\newfloat{codelisting}{h}{lop}}{\newfloat{codelisting}{h}{lop}[chapter]}
\floatname{codelisting}{Listing}
\newcommand*\listoflistings{\listof{codelisting}{List of Listings}}
\makeatother
\makeatletter
\@ifpackageloaded{caption}{}{\usepackage{caption}}
\@ifpackageloaded{subcaption}{}{\usepackage{subcaption}}
\makeatother
\makeatletter
\@ifpackageloaded{tcolorbox}{}{\usepackage[many]{tcolorbox}}
\makeatother
\makeatletter
\@ifundefined{shadecolor}{\definecolor{shadecolor}{rgb}{.97, .97, .97}}
\makeatother
\makeatletter
\makeatother
\ifLuaTeX
  \usepackage{selnolig}  % disable illegal ligatures
\fi
\IfFileExists{bookmark.sty}{\usepackage{bookmark}}{\usepackage{hyperref}}
\IfFileExists{xurl.sty}{\usepackage{xurl}}{} % add URL line breaks if available
\urlstyle{same} % disable monospaced font for URLs
\hypersetup{
  pdftitle={Conditionals},
  pdfauthor={Phil 296 - November 16},
  colorlinks=true,
  linkcolor={blue},
  filecolor={Maroon},
  citecolor={Blue},
  urlcolor={Blue},
  pdfcreator={LaTeX via pandoc}}

\title{Conditionals}
\author{Phil 296 - November 16}
\date{}

\begin{document}
\maketitle
\ifdefined\Shaded\renewenvironment{Shaded}{\begin{tcolorbox}[breakable, sharp corners, enhanced, interior hidden, borderline west={3pt}{0pt}{shadecolor}, frame hidden, boxrule=0pt]}{\end{tcolorbox}}\fi

Start with three questions:

\begin{enumerate}
\def\labelenumi{\arabic{enumi}.}
\tightlist
\item
  What do ordinary utterances of the form ``If this had happened, that
  would have happened'' mean? When are they true?
\item
  Given the role that sentences like those play in a lot of important
  tasks, e.g, assigning legal responsibility, planning complex actions,
  evaluating past actions, explaining and understand events, what do
  those sentences mean in those particular legal/political/scientific
  contexts?
\item
  What are some simple mathematical structures that yield accounts of
  what these sentences mean that are, at least roughly, like the answers
  to 1 and 2?
\end{enumerate}

And once we have an answer to 3, we can ask two more questions.

\begin{enumerate}
\def\labelenumi{\arabic{enumi}.}
\setcounter{enumi}{3}
\tightlist
\item
  How can we tinker with those models to make them even better answers
  to 1 and 2.
\item
  Can we use the models to resolve puzzle cases for 1 and 2 that we
  couldn't previously answer?
\end{enumerate}

This is a logic class, so we're going to mostly care about whether the
structures are `at least roughly' right in terms of which arguments
should or should not be valid.

\hypertarget{model-1---material-implication}{%
\subsection{Model 1 - Material
Implication}\label{model-1---material-implication}}

If \(A\) were true, \(B\) would be true just means \(A \supset B\). This
leads to absurd results. Think about the meme that is ``This is the
world we would have had if \ldots{}''.

\includegraphics[width=0.5\textwidth,height=\textheight]{future.jpg}

On the material implication account, the implicit conditional there is
true every single time. After all, in the meme versions, the \(A\) is
always false, and \(A \supset B\) is true whenever \(A\) is false. Since
that is not the world we would have had if, for example, we'd encouraged
everyone to get COVID as quickly as possible, this theory is false.

\hypertarget{model-2---strict-implication}{%
\subsection{Model 2 - Strict
Implication}\label{model-2---strict-implication}}

This is more plausible. It says that If \(A\) were true, \(B\) would be
true just means \(\Box(A \supset B)\). This is sometimes written with a
fishhook between \(A\) and \(B\), but I don't have that font on my
computer, so I won't write it here!

First problem, this seems to make basically all of these conditionals
come out false. It seems true that if I'd saved more money, I'd have
more money now. But there's a possible world where there is a malevolent
demon who would have incinerated my money if I'd saved more.

Well maybe we can solve that by saying one of the two following things.

\begin{itemize}
\tightlist
\item
  The \(\Box\) here is connected to a modal logic where the accessible
  worlds are the `normal' worlds, worlds that don't have demons, etc.
\item
  What the conditional really says is \(\Box((A \wedge N) \supset B)\),
  where N means that things are normal.
\end{itemize}

I won't repeat the proof here, but these approaches turn out to be
equivalent.

This view implies that the following three implications are always
valid. (Note I'll use \(\rightarrow\) here for the English `if \ldots{}
then', and all other symbols will be bits of mathematical notation that
should get

\begin{description}
\tightlist
\item[Antecedent Strengthening]
\(A \rightarrow B \vdash (A \wedge C) \rightarrow B\)
\item[Transitivity]
\(A \rightarrow B, B \rightarrow C \vdash A \rightarrow C\)
\item[Contraposition]
\(A \rightarrow B \vdash \neg B \rightarrow \neg A\)
\end{description}

And there are a lot of cases in English where these all seem to fail.
Note that the first two are connected. Assume that all instances of
transitivity are correct. Then the following argument is going to be
good.

\begin{itemize}
\tightlist
\item
  \((A \wedge C) \rightarrow A, A \rightarrow B \vdash (A \wedge C) \rightarrow B\)
\end{itemize}

But since the first premise is (surely!) a logical truth, we can drop
it, so this argument is valid.

\begin{itemize}
\tightlist
\item
  \(A \rightarrow B \vdash (A \wedge C) \rightarrow B\)
\end{itemize}

And that's just antecedent strengthening.

\hypertarget{model-3---normal-a-worlds}{%
\subsection{Model 3 - Normal A-worlds}\label{model-3---normal-a-worlds}}

Stipulate that we have a new function \(N\) from propositions to
propositions. So if \(A\) is a proposition, so is \(N(A)\). Intuitively,
\(N(A)\) means that \(A\) is true in a `normal' way, but we'll have to
build that intuition into the definition of \(N\) somehow.

Hypothesis: If \(A\) were true, \(B\) would be true just means
\(\Box(N(A) \supset B)\). So it means that B is true in all normal
A-worlds.

Good news: This means that the three problem inferences: Antecedent
Strengthening, Transitivity, and Contraposition all fail.

Bad news: Literally nothing can be shown to hold so far, since we've
said literally nothing about \(N\). (Well, that's a bit strong, since
\(\Box B \vdash A \rightarrow B\) is valid, but that's about it.)

Move: Tinker with the definition of \(N\) so that it does the job we
want. Since \(N\) lets in too much right now, it lets for example the
normal \(A\)-worlds be the worlds where Barack Obama teaches this class,
the obvious move is to add \textbf{constraints} on \(N\), restricting
what \(N\) could be mathematically.

To express these constraints, we'll do explicitly something I've been
doing implicitly a lot so far, identifying propositions with the sets of
worlds where they are true.

First constraint: \(N(A) \subseteq A\). That is, all worlds where
\(N(A)\) holds are worlds where \(A\) holds. This was originally part of
the intuitive conception, but we hadn't fixed it. This is sufficient
(and kind of necessary) to guarantee that \(A \rightarrow A\) is a
logical truth.

Second constraint: Let @ be a special name for the actual world. The
constraint is if \(@ \in A\), then \(@ \in N(A)\). If the actual world
is an \(A\)-world, it is a normal \(A\)-world.

I mentioned in class a strengthening of that constraint: if \(@ \in A\),
then \(\{@\} = N(A)\). That is, if the actual world is an \(A\)-world,
then it is the \textbf{only} normal \(A\)-world. How would things be if
Michigan had won last Saturday? They would be exactly like they are, in
\textbf{every} respect. We'll come back to this constraint later
(probably next week).

This gives us a logic, but it's still a very weak one. Note that it puts
no constraints on how normality relates different propositions. Imagine
that all the following are true.

\begin{itemize}
\tightlist
\item
  \(A \rightarrow B\), i.e., the normal \(A\)-worlds are \(B\)-worlds.
\item
  \(B \rightarrow A\), i.e., the normal \(B\)-worlds are \(A\)-worlds.
\item
  \(A \rightarrow \neg C\), i.e., no normal \(A\)-worlds are
  \(C\)-worlds.
\item
  \(B \rightarrow \neg C\), i.e., no normal \(B\)-worlds are
  \(C\)-worlds.
\end{itemize}

It feels like that should be enough to say that
\((A \wedge B) \rightarrow \neg C\). The normal \(A \wedge B\) worlds
just are some of those normal \(A\)-worlds and \(B\)-worlds. But nothing
we've said so far puts any constraints whatsoever on how \(N\) behaves
with respect to different inputs. So actually these are all consistent
with \((A \wedge B) \rightarrow \neg C\) being false, and even with
\((A \wedge B) \rightarrow C\) being true.

The constraints that Priest numbers 3 through 5 are designed to solve
problems like this one, but by this stage I think the idea that we have
any kind of simple, elegant, model has gone out the window. So let's try
a different approach.

\hypertarget{model-4---distances}{%
\subsection{Model 4 - Distances}\label{model-4---distances}}

Assume that there is a function \(d\) from pairs of worlds to distances.
For now assume distances are non-negative integers. (We'll come back to
this assumption maybe.) So read \(d(w_1, w_2) = n\) as the distance
between \(w_1\) and \(w_2\) is \(n\).

Say that a \textbf{sphere} of worlds around \(w\) is a set consisting of
all and only the worlds that are a particular distance from \(w\). So
sphere \(S_1\) is the set of all worlds that are distance 1 from \(w\).
For any world there will be a different set of spheres centered on it.

Here's the new proposed truth conditions for \(A \rightarrow B\). They
are disjunctive. The sentence is true, at \(w\), if either of the
following two conditions are met:

\begin{enumerate}
\def\labelenumi{\arabic{enumi}.}
\tightlist
\item
  There are no worlds where \(A\) is true; or
\item
  At the nearest sphere that contains any \(A\)-worlds, all \(A\)-worlds
  are \(B\)-worlds.
\end{enumerate}

As the other handout shows, this approach yields counterexamples to
Antecedent Strengthening, Transitivity, and Contraposition. But it makes
some related things come out true.

\begin{description}
\tightlist
\item[Modified Transitivity]
A \rightarrow B, (A \wedge B) \rightarrow C \vdash A \rightarrow C
\end{description}

This is important because, coming back to question 2, we really do seem
to want something like transitivity when evaluating how policies worked.
Imagine someone who argues that we should have had stricter mask
mandates in 2020 arguing as follows:

\begin{enumerate}
\def\labelenumi{\arabic{enumi}.}
\tightlist
\item
  If we'd had stricter mandates, we'd have had fewer cases.
\item
  If we'd had fewer cases, fewer people would have died.
\item
  So if we'd had stricter mandates, fewer people would have died.
\end{enumerate}

You can quibble on the premises - there is still a big debate about
whether mandates work and so whether premise 1 is true - but the
reasoning seems kind of ok. It would be weird for a mandate opponent to
say, yes those premises are fine, but they don't support your
conclusion. So what's gone on here? One hypothesis is that both
supporters and opponents of the argument are really reaading the second
premise as this:

\begin{itemize}
\tightlist
\item
  If we'd had \textbf{mask mandates and} fewer cases, fewer people would
  have died.
\end{itemize}

And now the argument is valid, and we can go back to what seems like the
interesting policy question: whether premise 1 is actually true.

\hypertarget{modifying-model-4}{%
\subsection{Modifying Model 4}\label{modifying-model-4}}

If you like this model so far, then you might want to tinker with it.
The natural way to `tinker' is to put constraints on \(d\). Like with
\(N\), we introduced this as a function, and with an intended
interpretation, but we didn't do the hard work of showing that the
function really satisfied the intended interpretation. Just like with
modal logic, and with \(N\), the more constraints you put on \(d\), in
general the more things you can prove. Here are some constraints that
seem natural.

\begin{itemize}
\tightlist
\item
  \(d(w, w) = 0\)
\end{itemize}

That is, every world is distance 0 from itself. Without that, we
actually don't have \(A \rightarrow B, A \vdash B\). It could be that
the `nearest' \(A\)-worlds do not include the actual world. This seems
like a very natural constraint on \(d\).

Here are two more constraints that make sense given the `distance'
metaphor, but as far as I know make no difference to the logic.

\begin{itemize}
\tightlist
\item
  \(d(w_1, w_2) = d(w_2, w_1)\)
\item
  \(d(w_1, w_2) + d(w_2, w_3) \leq d(w_1, w_3)\)
\end{itemize}

A more interesting constraint, also suggested by the distance metaphor,
is this one:

\begin{itemize}
\tightlist
\item
  If \(w_1 \neq w2\), then \(d(w_1, w_2) > 0\)
\end{itemize}

That is, all distinct worlds have positive distance between them. If
that's right, we get the following as a valid argument schema.

\begin{itemize}
\tightlist
\item
  \(A, B \vdash A \rightarrow B\)
\end{itemize}

Why? Assume \(A, B\) are true at a particular world \(w\). The sphere of
nearest \(A\)-worlds will just be \(\{w\}\), since it is distance 0
away, and all other worlds are positive distance away. And \(B\) is true
at all of the worlds in \(S_0\).

\hypertarget{conditional-excluded-middle}{%
\subsection{Conditional Excluded
Middle}\label{conditional-excluded-middle}}

I don't think we'll get to this today, but I wanted to note that one of
the big debates that comes next concerns the following two (equivalent)
principles:

\begin{itemize}
\tightlist
\item
  \(\vdash (A \rightarrow B) \vee (A \rightarrow \neg B)\)
\item
  \(\vdash (A \rightarrow \neg B) \equiv \neg (A \rightarrow B)\)
\end{itemize}

Should those be true? Actually, going back to our original question, we
can separate those out - should those be true in ordinary language, and
should they be true in the artificial language we use for formally
modeling decisions like mask mandates? Maybe those get different
answers.

Anyway, they don't hold in any of the last three models we looked at, so
if we want to make them hold, we need to tinker some more. If we have
time, we'll look at the tinkering that is needed.



\end{document}
