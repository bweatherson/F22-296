% Options for packages loaded elsewhere
\PassOptionsToPackage{unicode}{hyperref}
\PassOptionsToPackage{hyphens}{url}
%
\documentclass[
  14pt,
  letterpaper,
  ignorenonframetext,
  aspectratio=169,
]{beamer}
\usepackage{pgfpages}
\setbeamertemplate{caption}[numbered]
\setbeamertemplate{caption label separator}{: }
\setbeamercolor{caption name}{fg=normal text.fg}
\beamertemplatenavigationsymbolsempty
% Prevent slide breaks in the middle of a paragraph
\widowpenalties 1 10000
\raggedbottom
\setbeamertemplate{part page}{
  \centering
  \begin{beamercolorbox}[sep=16pt,center]{part title}
    \usebeamerfont{part title}\insertpart\par
  \end{beamercolorbox}
}
\setbeamertemplate{section page}{
  \centering
  \begin{beamercolorbox}[sep=12pt,center]{part title}
    \usebeamerfont{section title}\insertsection\par
  \end{beamercolorbox}
}
\setbeamertemplate{subsection page}{
  \centering
  \begin{beamercolorbox}[sep=8pt,center]{part title}
    \usebeamerfont{subsection title}\insertsubsection\par
  \end{beamercolorbox}
}
\AtBeginPart{
  \frame{\partpage}
}
\AtBeginSection{
  \ifbibliography
  \else
    \frame{\sectionpage}
  \fi
}
\AtBeginSubsection{
  \frame{\subsectionpage}
}

\usepackage{amsmath,amssymb}
\usepackage{lmodern}
\usepackage{iftex}
\ifPDFTeX
  \usepackage[T1]{fontenc}
  \usepackage[utf8]{inputenc}
  \usepackage{textcomp} % provide euro and other symbols
\else % if luatex or xetex
  \usepackage{unicode-math}
  \defaultfontfeatures{Scale=MatchLowercase}
  \defaultfontfeatures[\rmfamily]{Ligatures=TeX,Scale=1}
  \setmainfont[Scale=MatchLowercase]{SF Pro Text Light}
  \setmathfont[]{STIX Two Math}
\fi
\usetheme[]{boxes}
\usecolortheme{wolverine}
\usefonttheme{serif} % use mainfont rather than sansfont for slide text
% Use upquote if available, for straight quotes in verbatim environments
\IfFileExists{upquote.sty}{\usepackage{upquote}}{}
\IfFileExists{microtype.sty}{% use microtype if available
  \usepackage[]{microtype}
  \UseMicrotypeSet[protrusion]{basicmath} % disable protrusion for tt fonts
}{}
\makeatletter
\@ifundefined{KOMAClassName}{% if non-KOMA class
  \IfFileExists{parskip.sty}{%
    \usepackage{parskip}
  }{% else
    \setlength{\parindent}{0pt}
    \setlength{\parskip}{6pt plus 2pt minus 1pt}}
}{% if KOMA class
  \KOMAoptions{parskip=half}}
\makeatother
\usepackage{xcolor}
\newif\ifbibliography
\setlength{\emergencystretch}{3em} % prevent overfull lines
\setcounter{secnumdepth}{-\maxdimen} % remove section numbering


\providecommand{\tightlist}{%
  \setlength{\itemsep}{0pt}\setlength{\parskip}{0pt}}\usepackage{longtable,booktabs,array}
\usepackage{calc} % for calculating minipage widths
\usepackage{caption}
% Make caption package work with longtable
\makeatletter
\def\fnum@table{\tablename~\thetable}
\makeatother
\usepackage{graphicx}
\makeatletter
\def\maxwidth{\ifdim\Gin@nat@width>\linewidth\linewidth\else\Gin@nat@width\fi}
\def\maxheight{\ifdim\Gin@nat@height>\textheight\textheight\else\Gin@nat@height\fi}
\makeatother
% Scale images if necessary, so that they will not overflow the page
% margins by default, and it is still possible to overwrite the defaults
% using explicit options in \includegraphics[width, height, ...]{}
\setkeys{Gin}{width=\maxwidth,height=\maxheight,keepaspectratio}
% Set default figure placement to htbp
\makeatletter
\def\fps@figure{htbp}
\makeatother

\captionsetup[figure]{labelformat=empty}
\usepackage{pgfpages}
\setbeamertemplate{itemize item}[circle]
\setbeamertemplate{footline}[frame number]{}
\mode<handout>{\pgfpagesuselayout{6 on 1}[letterpaper, border shrink=8mm]}
\AtBeginSection{%
   \begin{frame}
       \tableofcontents[currentsection]
   \end{frame}
}
\usepackage{tikz}
\usetikzlibrary{positioning,arrows,calc}
\tikzset{
  modal/.style={>=stealth',
    shorten >=1pt,
    shorten <=1pt,
    auto,
   node distance=1.5cm,
   label distance=2pt,
   semithick},
 every label/.style={phantom,align=left},
 world/.style = {circle,draw,minimum size=0.5cm,fill=gray!15},
 modal every node/.style={world},
 point/.style={circle,draw,inner sep=0.5mm,fill=black},
 phantom/.style={rectangle,inner sep=0pt,draw=none,fill=none},
 reflexive above/.style={->,loop,looseness=7,in=60,out=120},
 reflexive below/.style={->,loop,looseness=7,in=240,out=300},
 reflexive left/.style={->,loop,looseness=7,in=150,out=210},
 reflexive right/.style={->,loop,looseness=7,in=30,out=330}}
\makeatletter
\makeatother
\makeatletter
\makeatother
\makeatletter
\@ifpackageloaded{caption}{}{\usepackage{caption}}
\AtBeginDocument{%
\ifdefined\contentsname
  \renewcommand*\contentsname{Table of contents}
\else
  \newcommand\contentsname{Table of contents}
\fi
\ifdefined\listfigurename
  \renewcommand*\listfigurename{List of Figures}
\else
  \newcommand\listfigurename{List of Figures}
\fi
\ifdefined\listtablename
  \renewcommand*\listtablename{List of Tables}
\else
  \newcommand\listtablename{List of Tables}
\fi
\ifdefined\figurename
  \renewcommand*\figurename{Figure}
\else
  \newcommand\figurename{Figure}
\fi
\ifdefined\tablename
  \renewcommand*\tablename{Table}
\else
  \newcommand\tablename{Table}
\fi
}
\@ifpackageloaded{float}{}{\usepackage{float}}
\floatstyle{ruled}
\@ifundefined{c@chapter}{\newfloat{codelisting}{h}{lop}}{\newfloat{codelisting}{h}{lop}[chapter]}
\floatname{codelisting}{Listing}
\newcommand*\listoflistings{\listof{codelisting}{List of Listings}}
\makeatother
\makeatletter
\@ifpackageloaded{caption}{}{\usepackage{caption}}
\@ifpackageloaded{subcaption}{}{\usepackage{subcaption}}
\makeatother
\makeatletter
\@ifpackageloaded{tcolorbox}{}{\usepackage[many]{tcolorbox}}
\makeatother
\makeatletter
\@ifundefined{shadecolor}{\definecolor{shadecolor}{rgb}{.97, .97, .97}}
\makeatother
\makeatletter
\makeatother
\ifLuaTeX
  \usepackage{selnolig}  % disable illegal ligatures
\fi
\IfFileExists{bookmark.sty}{\usepackage{bookmark}}{\usepackage{hyperref}}
\IfFileExists{xurl.sty}{\usepackage{xurl}}{} % add URL line breaks if available
\urlstyle{same} % disable monospaced font for URLs
\hypersetup{
  pdftitle={Honors Logic, Lecture 13 - Modal Logic},
  pdfauthor={Brian Weatherson},
  hidelinks,
  pdfcreator={LaTeX via pandoc}}

\title{Honors Logic, Lecture 13 - Modal Logic}
\author{Brian Weatherson}
\date{2022-10-12}

\begin{document}
\frame{\titlepage}
\ifdefined\Shaded\renewenvironment{Shaded}{\begin{tcolorbox}[borderline west={3pt}{0pt}{shadecolor}, frame hidden, boxrule=0pt, enhanced, interior hidden, breakable, sharp corners]}{\end{tcolorbox}}\fi

\begin{frame}{Six New Steps}
\protect\hypertarget{six-new-steps}{}
\begin{enumerate}
\tightlist
\item
  Every line has a world number.
\item
  The rules for non-modal connectives preserve world.
\item
  For negated modals, move negation inside and flip
\item
  For true \(\Diamond\) sentences, introduce a new world.
\item
  For true \(\Box\) worlds, do nothing at first, but make boxed sentence
  true everywhere accessible.
\item
  Only close a branch when a sentence is true and false at same world.
\end{enumerate}
\end{frame}

\begin{frame}{Step 1}
\protect\hypertarget{step-1}{}
\begin{enumerate}
\tightlist
\item
  \textbf{Every line has a world number.}
\item
  The rules for non-modal connectives preserve world.
\item
  For negated modals, move negation inside and flip
\item
  For true \(\Diamond\) sentences, introduce a new world.
\item
  For true \(\Box\) worlds, do nothing at first, but make boxed sentence
  true everywhere accessible.
\item
  Only close a branch when a sentence is true and false at same world.
\end{enumerate}
\end{frame}

\begin{frame}{World Numbers}
\protect\hypertarget{world-numbers}{}
Lines now look like this.

\begin{center}
$p \wedge q, 1$
\end{center}

Read this as saying that the conjunction \(p \wedge q\) is true at world
1.
\end{frame}

\begin{frame}{Step 2}
\protect\hypertarget{step-2}{}
\begin{enumerate}
\tightlist
\item
  Every line has a world number.
\item
  \textbf{The rules for non-modal connectives preserve world.}
\item
  For negated modals, move negation inside and flip
\item
  For true \(\Diamond\) sentences, introduce a new world.
\item
  For true \(\Box\) worlds, do nothing at first, but make boxed sentence
  true everywhere accessible.
\item
  Only close a branch when a sentence is true and false at same world.
\end{enumerate}
\end{frame}

\begin{frame}{World Preservation}
\protect\hypertarget{world-preservation}{}
\begin{itemize}
\tightlist
\item
  All the old rules didn't have line numbers.
\item
  But the way to apply them is just to keep the world numbers the same.
\end{itemize}
\end{frame}

\begin{frame}{Example 1}
\protect\hypertarget{example-1}{}
\begin{center}
$p \wedge q, 3$ \\
$p, 3$ \\
$q, 3$ \\
\end{center}
\end{frame}

\begin{frame}{Example 2}
\protect\hypertarget{example-2}{}
\begin{center}
$p \supset q, 4$ \\
$\neg p, 4               q, 4$ \\
\end{center}
\end{frame}

\begin{frame}{Step 3}
\protect\hypertarget{step-3}{}
\begin{enumerate}
\tightlist
\item
  Every line has a world number.
\item
  The rules for non-modal connectives preserve world.
\item
  \textbf{For negated modals, move negation inside and flip.}
\item
  For true \(\Diamond\) sentences, introduce a new world.
\item
  For true \(\Box\) worlds, do nothing at first, but make boxed sentence
  true everywhere accessible.
\item
  Only close a branch when a sentence is true and false at same world.
\end{enumerate}
\end{frame}

\begin{frame}{Negated Modals (\(\Diamond\))}
\protect\hypertarget{negated-modals-diamond}{}
For each of them, the rule is move the negation inside, and invert.

\begin{center}
$\neg \Diamond A, n$ \\
$\Box \neg A, n$
\end{center}

Note that the world stays the same, as does what comes after the modal.
\end{frame}

\begin{frame}{Negated Modals (\(\Box\))}
\protect\hypertarget{negated-modals-box}{}
For each of them, the rule is move the negation inside, and invert.

\begin{center}
$\neg \Box A, n$ \\
$\Diamond \neg A, n$
\end{center}

Note that the world stays the same, as does what comes after the modal.
\end{frame}

\begin{frame}{Step 4}
\protect\hypertarget{step-4}{}
\begin{enumerate}
\tightlist
\item
  Every line has a world number.
\item
  The rules for non-modal connectives preserve world.
\item
  For negated modals, move negation inside and flip.
\item
  \textbf{For true \(\Diamond\) sentences, introduce a new world.}
\item
  For true \(\Box\) worlds, do nothing at first, but make boxed sentence
  true everywhere accessible.
\item
  Only close a branch when a sentence is true and false at same world.
\end{enumerate}
\end{frame}

\begin{frame}{Example 3}
\protect\hypertarget{example-3}{}
Here is an instance of the true \(\Diamond\) rule in action.

\begin{center}
$\Diamond(p \wedge \Box q), 4$ \\
$4r5$ \\
$p \wedge \Box q, 5$
\end{center}

\begin{itemize}
\tightlist
\item
  This would only be ok if 5 had not been used on the branch before.
\end{itemize}
\end{frame}

\begin{frame}{General Rule}
\protect\hypertarget{general-rule}{}
When you have a true \(Diamond\) sentence:

\begin{itemize}
\tightlist
\item
  On a new line, copy down the sentence;
\item
  Delete the \(\Diamond\);
\item
  Change the world number to a number that didn't previously appear on
  the tree.
\item
  Write that the world from the original sentence can access the new
  world.
\item
  That's it; there are no more rules to apply.
\end{itemize}
\end{frame}

\begin{frame}{Explanation}
\protect\hypertarget{explanation}{}
A true \(\Diamond\) sentence says that at some accessible world, what's
inside the \(\Diamond\) is true.

\begin{itemize}
\tightlist
\item
  Since the world names are arbitrary, we're just giving whatever world
  that is an arbitrary name.
\item
  And it's accessible, so we say that the original world can see it.
\item
  You have two lines to write down; the order you write them in doesn't
  matter.
\end{itemize}
\end{frame}

\begin{frame}{Step 5}
\protect\hypertarget{step-5}{}
\begin{enumerate}
\tightlist
\item
  Every line has a world number.
\item
  The rules for non-modal connectives preserve world.
\item
  For negated modals, move negation inside and flip.
\item
  For true \(\Diamond\) sentences, introduce a new world.
\item
  \textbf{For true \(\Box\) worlds, do nothing at first, but make boxed
  sentence true everywhere accessible.}
\item
  Only close a branch when a sentence is true and false at same world.
\end{enumerate}
\end{frame}

\begin{frame}{Do Nothing}
\protect\hypertarget{do-nothing}{}
Here is a completed tableau showing that \(\Box p \vdash p\) is not a
theorem of the basic modal logic K.

\begin{center}
$\Box p, 0$ \\
$\neg p, 0$
\end{center}

There's nothing more to do.
\end{frame}

\begin{frame}{Example 4 - \(\Box p \vdash \Box \Box p\)}
\protect\hypertarget{example-4---box-p-vdash-box-box-p}{}
\begin{center}
$\Box p, 0$ \\
$\neg \Box \Box p, 0$ \\
$\Diamond \neg \Box p, 0$ \\
$0r1$ \\
$\neg \Box p, 1$ \\
$p, 1$ \\ 
$\Diamond \neg p, 1$ \\
$1r2$ \\
$\neg p, 2$
\end{center}

All the rules are applied. Crucially, because the only 0rx is for x=1,
just apply line 1 to world 1.
\end{frame}

\begin{frame}{Step 6}
\protect\hypertarget{step-6}{}
\begin{enumerate}
\tightlist
\item
  Every line has a world number.
\item
  The rules for non-modal connectives preserve world.
\item
  For negated modals, move negation inside and flip.
\item
  For true \(\Diamond\) sentences, introduce a new world.
\item
  For true \(\Box\) worlds, do nothing at first, but make boxed sentence
  true everywhere accessible.
\item
  \textbf{Only close a branch when a sentence is true and false at same
  world.}
\end{enumerate}
\end{frame}

\begin{frame}{Don't do this!!!}
\protect\hypertarget{dont-do-this}{}
A tableau that `shows' the mistaken claim
\(\vdash \neg(\Diamond p \wedge \Diamond \neg p)\)

\begin{center}
$\neg \neg(\Diamond p \wedge \Diamond \neg p), 0$ \\
$\Diamond p \wedge \Diamond \neg p, 0$ \\
$\Diamond p, 0$ \\
$\Diamond \neg p, 0$ \\
$0r1$ \\
$p, 1$ \\
$0r2$ \\
$\neg p, 2$ \\
x (since $p$ and $\neg p$)
\end{center}
\end{frame}

\begin{frame}{More examples}
\protect\hypertarget{more-examples}{}
We'll work through some more examples from the exercises at the end of
chapter 2
\end{frame}



\end{document}
