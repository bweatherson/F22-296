% Options for packages loaded elsewhere
\PassOptionsToPackage{unicode}{hyperref}
\PassOptionsToPackage{hyphens}{url}
%
\documentclass[
  14pt,
  letterpaper,
  ignorenonframetext,
  handout]{beamer}
\usepackage{pgfpages}
\setbeamertemplate{caption}[numbered]
\setbeamertemplate{caption label separator}{: }
\setbeamercolor{caption name}{fg=normal text.fg}
\beamertemplatenavigationsymbolsempty
% Prevent slide breaks in the middle of a paragraph
\widowpenalties 1 10000
\raggedbottom
\setbeamertemplate{part page}{
  \centering
  \begin{beamercolorbox}[sep=16pt,center]{part title}
    \usebeamerfont{part title}\insertpart\par
  \end{beamercolorbox}
}
\setbeamertemplate{section page}{
  \centering
  \begin{beamercolorbox}[sep=12pt,center]{part title}
    \usebeamerfont{section title}\insertsection\par
  \end{beamercolorbox}
}
\setbeamertemplate{subsection page}{
  \centering
  \begin{beamercolorbox}[sep=8pt,center]{part title}
    \usebeamerfont{subsection title}\insertsubsection\par
  \end{beamercolorbox}
}
\AtBeginPart{
  \frame{\partpage}
}
\AtBeginSection{
  \ifbibliography
  \else
    \frame{\sectionpage}
  \fi
}
\AtBeginSubsection{
  \frame{\subsectionpage}
}

\usepackage{amsmath,amssymb}
\usepackage{lmodern}
\usepackage{iftex}
\ifPDFTeX
  \usepackage[T1]{fontenc}
  \usepackage[utf8]{inputenc}
  \usepackage{textcomp} % provide euro and other symbols
\else % if luatex or xetex
  \usepackage{unicode-math}
  \defaultfontfeatures{Scale=MatchLowercase}
  \defaultfontfeatures[\rmfamily]{Ligatures=TeX,Scale=1}
  \setmainfont[BoldFont = SF Pro Text Semibold, Scale =
MatchLowercase]{SF Pro Text Light}
  \setmathfont[]{STIX Two Math}
\fi
\usetheme[]{boxes}
\usecolortheme{wolverine}
\usefonttheme{serif} % use mainfont rather than sansfont for slide text
% Use upquote if available, for straight quotes in verbatim environments
\IfFileExists{upquote.sty}{\usepackage{upquote}}{}
\IfFileExists{microtype.sty}{% use microtype if available
  \usepackage[]{microtype}
  \UseMicrotypeSet[protrusion]{basicmath} % disable protrusion for tt fonts
}{}
\makeatletter
\@ifundefined{KOMAClassName}{% if non-KOMA class
  \IfFileExists{parskip.sty}{%
    \usepackage{parskip}
  }{% else
    \setlength{\parindent}{0pt}
    \setlength{\parskip}{6pt plus 2pt minus 1pt}}
}{% if KOMA class
  \KOMAoptions{parskip=half}}
\makeatother
\usepackage{xcolor}
\newif\ifbibliography
\setlength{\emergencystretch}{3em} % prevent overfull lines
\setcounter{secnumdepth}{-\maxdimen} % remove section numbering


\providecommand{\tightlist}{%
  \setlength{\itemsep}{0pt}\setlength{\parskip}{0pt}}\usepackage{longtable,booktabs,array}
\usepackage{calc} % for calculating minipage widths
\usepackage{caption}
% Make caption package work with longtable
\makeatletter
\def\fnum@table{\tablename~\thetable}
\makeatother
\usepackage{graphicx}
\makeatletter
\def\maxwidth{\ifdim\Gin@nat@width>\linewidth\linewidth\else\Gin@nat@width\fi}
\def\maxheight{\ifdim\Gin@nat@height>\textheight\textheight\else\Gin@nat@height\fi}
\makeatother
% Scale images if necessary, so that they will not overflow the page
% margins by default, and it is still possible to overwrite the defaults
% using explicit options in \includegraphics[width, height, ...]{}
\setkeys{Gin}{width=\maxwidth,height=\maxheight,keepaspectratio}
% Set default figure placement to htbp
\makeatletter
\def\fps@figure{htbp}
\makeatother

\captionsetup[figure]{labelformat=empty}
\usepackage{pgfpages}
\setbeamertemplate{itemize item}[circle]
\setbeamertemplate{footline}[frame number]{}
\mode<handout>{\pgfpagesuselayout{6 on 1}[letterpaper, border shrink=8mm]}
\AtBeginSection{%
   \begin{frame}
       \tableofcontents[currentsection]
   \end{frame}
}
\usepackage{tikz}
\usetikzlibrary{positioning,arrows,calc}
\tikzset{
  modal/.style={>=stealth',
    shorten >=1pt,
    shorten <=1pt,
    auto,
   node distance=1.5cm,
   label distance=2pt,
   semithick},
 every label/.style={phantom,align=left},
 world/.style = {circle,draw,minimum size=0.5cm,fill=gray!15},
 modal every node/.style={world},
 point/.style={circle,draw,inner sep=0.5mm,fill=black},
 phantom/.style={rectangle,inner sep=0pt,draw=none,fill=none},
 reflexive above/.style={->,loop,looseness=7,in=60,out=120},
 reflexive below/.style={->,loop,looseness=7,in=240,out=300},
 reflexive left/.style={->,loop,looseness=7,in=150,out=210},
 reflexive right/.style={->,loop,looseness=7,in=30,out=330}}
\makeatletter
\makeatother
\makeatletter
\makeatother
\makeatletter
\@ifpackageloaded{caption}{}{\usepackage{caption}}
\AtBeginDocument{%
\ifdefined\contentsname
  \renewcommand*\contentsname{Table of contents}
\else
  \newcommand\contentsname{Table of contents}
\fi
\ifdefined\listfigurename
  \renewcommand*\listfigurename{List of Figures}
\else
  \newcommand\listfigurename{List of Figures}
\fi
\ifdefined\listtablename
  \renewcommand*\listtablename{List of Tables}
\else
  \newcommand\listtablename{List of Tables}
\fi
\ifdefined\figurename
  \renewcommand*\figurename{Figure}
\else
  \newcommand\figurename{Figure}
\fi
\ifdefined\tablename
  \renewcommand*\tablename{Table}
\else
  \newcommand\tablename{Table}
\fi
}
\@ifpackageloaded{float}{}{\usepackage{float}}
\floatstyle{ruled}
\@ifundefined{c@chapter}{\newfloat{codelisting}{h}{lop}}{\newfloat{codelisting}{h}{lop}[chapter]}
\floatname{codelisting}{Listing}
\newcommand*\listoflistings{\listof{codelisting}{List of Listings}}
\makeatother
\makeatletter
\@ifpackageloaded{caption}{}{\usepackage{caption}}
\@ifpackageloaded{subcaption}{}{\usepackage{subcaption}}
\makeatother
\makeatletter
\@ifpackageloaded{tcolorbox}{}{\usepackage[many]{tcolorbox}}
\makeatother
\makeatletter
\@ifundefined{shadecolor}{\definecolor{shadecolor}{rgb}{.97, .97, .97}}
\makeatother
\makeatletter
\makeatother
\ifLuaTeX
  \usepackage{selnolig}  % disable illegal ligatures
\fi
\IfFileExists{bookmark.sty}{\usepackage{bookmark}}{\usepackage{hyperref}}
\IfFileExists{xurl.sty}{\usepackage{xurl}}{} % add URL line breaks if available
\urlstyle{same} % disable monospaced font for URLs
\hypersetup{
  pdftitle={Honors Logic, Lecture 18 - Nature of Epistemic Logic},
  pdfauthor={Brian Weatherson},
  hidelinks,
  pdfcreator={LaTeX via pandoc}}

\title{Honors Logic, Lecture 18 - Nature of Epistemic Logic}
\author{Brian Weatherson}
\date{11/2/22}

\begin{document}
\frame{\titlepage}
\ifdefined\Shaded\renewenvironment{Shaded}{\begin{tcolorbox}[sharp corners, borderline west={3pt}{0pt}{shadecolor}, interior hidden, frame hidden, enhanced, breakable, boxrule=0pt]}{\end{tcolorbox}}\fi

\begin{frame}{Models}
\protect\hypertarget{models}{}
So far we've mostly talked about \textbf{models}, which have three
parts.

\begin{enumerate}
\tightlist
\item
  A set of worlds, which are really just featureless points.
\item
  An \(r\) relation on those worlds.
\item
  An assignment \(v\) of atomic sentences to worlds, intuitively meaning
  that the atomic is true at those points.
\end{enumerate}
\end{frame}

\begin{frame}{Frames}
\protect\hypertarget{frames}{}
A \textbf{frame} is just a model without the last part. That is, it just
has

\begin{enumerate}
\tightlist
\item
  A set of worlds.
\item
  An \(r\) relation on those worlds.
\end{enumerate}
\end{frame}

\begin{frame}{Truth on a Frame}
\protect\hypertarget{truth-on-a-frame}{}
Say that a sentence \(A\) is true on a frame \(\langle W, R \rangle\)
iff \textbf{any} way of adding a \(V\) and turning the frame into a
model will make \(A\) true at every point in the model.
\end{frame}

\begin{frame}{Example}
\protect\hypertarget{example}{}
\begin{itemize}
\tightlist
\item
  Consider the frame where the points are the real numbers, and \(xry\)
  holds iff \(x \geq y\).
\item
  And consider the sentence \(\Box p \supset p\).
\item
  Is that true on the frame?
\end{itemize}
\end{frame}

\begin{frame}{Example}
\protect\hypertarget{example-1}{}
Yes!

\begin{itemize}
\tightlist
\item
  It doesn't matter what set you assign to \(p\), at any point, either
  \(p\) is in the set or it isn't.
\item
  If it is, then \(\Box p \supset p\) is true at the point, since \(p\)
  is true.
\item
  If it is not, then \(\Box p\) is false at the point, since one point
  (namely that very point) that the point can see makes \(p\) false.
\item
  And that implies \(\Box p \supset p\) is true at the point.
\end{itemize}
\end{frame}

\begin{frame}{Another Example}
\protect\hypertarget{another-example}{}
Same frame, but consider now the sentence
\(\Box p \supset \Box \Box p\).

\begin{itemize}
\tightlist
\item
  For any point, we can consider the question, is \(p\) true at that
  point and all earlier points?
\item
  If so, \(\Box \Box p\) will be true, since anywhere you can get in two
  steps from the point will make \(p\) true.
\item
  If not, \(\Box p\) will be false, since somewhere you can get makes
  \(p\) false.
\item
  Either way, \(\Box p \supset \Box \Box p\) will be true.
\end{itemize}
\end{frame}

\begin{frame}{Characteristic Sentences}
\protect\hypertarget{characteristic-sentences}{}
Consider not just an individual frame, but a whole set of frames. A
sentence \(A\) \textbf{characterises} that set if the following two
things hold.

\begin{itemize}
\tightlist
\item
  At \textbf{any} model built from any frame in the set, \(A\) is true.
\item
  For any frame not in the set, there is \textbf{some} point in
  \textbf{some} model built on the frame where \(A\) is false.
\item
  So there is this claim about \(A\) that is true at \textbf{all and
  only} frames in the set.
\end{itemize}
\end{frame}

\begin{frame}{Charactistic Sentences}
\protect\hypertarget{charactistic-sentences}{}
I'm not going to go over the proofs of all these I think (unless we have
more time than I expect), but it turns out that there are quite a lot of
sets of frames that have characteristic sentences in this (rather
demanding) sense.
\end{frame}

\begin{frame}{Charactistic Sentences}
\protect\hypertarget{charactistic-sentences-1}{}
\begin{itemize}
\tightlist
\item
  The class of \textbf{reflexive} frames is characterised by
  \(\Box p \supset p\).
\item
  The class of \textbf{symmetric} frames is characterised by
  \(p \supset \Box \Diamond p\).
\item
  The class of \textbf{transitive} frames is characterised by
  \(\Box p \supset \Box \Box p\).
\item
  These can be mix and matched: The class of \textbf{reflexive,
  transitive} frames is characterised by
  \((\Box p \supset p) \wedge (\Box p \supset \Box \Box p)\).
\end{itemize}
\end{frame}

\begin{frame}{Philosophical Significance}
\protect\hypertarget{philosophical-significance}{}
These results mean we can go back and forth easily between thinking
about what sentences are logical truths (for a particular interpretation
of Box), and what properties the relevant possibility relation has.

\begin{itemize}
\tightlist
\item
  And this is useful, because sometimes it is much more intuitive to
  think about one than the other.
\end{itemize}
\end{frame}

\begin{frame}{Thinking about Possibility}
\protect\hypertarget{thinking-about-possibility}{}
When we think about what these models might really mean, the \(r\)
relation is a kind of possibility.

\begin{itemize}
\tightlist
\item
  \(wrx\) means that if \(w\) is how things actually are, then \(x\) is
  a way things might be, in the sense of might that we care about.
  \pause  The point of the results in chapter 4 is that we now have two
  ways to think about these possibility claims.
\end{itemize}

\begin{enumerate}
\tightlist
\item
  We can look at them directly, and this may tell us something about
  certain modal sentences.
\item
  Or we can look at the sentences, and that can tell us something about
  the possibility claims.
\end{enumerate}
\end{frame}

\begin{frame}{Epistemic Modality}
\protect\hypertarget{epistemic-modality}{}
I want to walk through a bit how this plays out in contemporary debates
in epistemology (the study of knowledge).

\begin{itemize}
\tightlist
\item
  We will treat \(\Box A\) as meaning \emph{Hero knows that A}.
\item
  So \(\Diamond A\) means \emph{For all hero knows, A}, or (roughly)
  \emph{A might be true (from Hero's perspective)}.
\item
  And \(wrx\) means \emph{If w is actual, then x is possible for Hero}.
  That is, if \(w\) is actual, then for all Hero knows, they are
  actually in \(x\).
\end{itemize}
\end{frame}

\begin{frame}{Reflexivity}
\protect\hypertarget{reflexivity}{}
Remember there is a tight connection between these two claims.

\begin{enumerate}
\tightlist
\item
  \(r\) is reflexive, i.e., \(wrw\) for all \(w\).
\item
  It is always true that \(\Box A \supset A\).
\end{enumerate}
\end{frame}

\begin{frame}{Reflexivity and Epistemic Modals}
\protect\hypertarget{reflexivity-and-epistemic-modals}{}
Both of these look like they should hold.

\begin{enumerate}
\tightlist
\item
  If Hero is in world \(w\), then for all she knows, she could be in
  world \(w \pause\).
\item
  If Hero knows that \(A\), then it is the case that \(A\).
\end{enumerate}
\end{frame}

\begin{frame}{Symmetry}
\protect\hypertarget{symmetry}{}
Remember there is a tight connection between these two claims.

\begin{enumerate}
\tightlist
\item
  \(r\) is symmetric, i.e., if \(wrx\) then \(xrw\).
\item
  It is always true that \(A \supset \Box \Diamond A \pause\).
\end{enumerate}

Question:

\begin{itemize}
\tightlist
\item
  Should this hold for the epistemic interpretation of \(\Box\)?
\end{itemize}
\end{frame}

\begin{frame}{Symmetry and Epistemic Modals}
\protect\hypertarget{symmetry-and-epistemic-modals}{}
This does not look particularly plausible. Let \(p\) be something that's
actually true, but which Hero believes is false. (Hero could be anyone,
so they could have false beliefs.) We'll start thinking directly about
the models.
\end{frame}

\begin{frame}{Symmetry and Epistemic Modals}
\protect\hypertarget{symmetry-and-epistemic-modals-1}{}
\begin{itemize}
\tightlist
\item
  Let \(w_a\) be the actual world, and \(w_b\) the world in which
  everything is exactly as Hero thinks it is (in all respects).
\item
  Then \(w_arw_b\), since \(w_b\) is surely possible for Hero from the
  actual perspective.
\item
  But if she were in \(w_b\), then \(w_a\) would not be possible,
  because she would know that \(p\) is false, and in \(w_a\) it is true.
\item
  So \(w_arw_b\) but not \(w_brw_a\), showing symmetry fails.
\end{itemize}
\end{frame}

\begin{frame}{Symmetry and Epistemic Modals (cont)}
\protect\hypertarget{symmetry-and-epistemic-modals-cont}{}
Now think about the axiom

\begin{itemize}
\tightlist
\item
  \(p \supset \Box \Diamond p\)
\end{itemize}

Remember Hero thinks \(p\) is false, but it's actually true. Is it the
case that \(\Box \Diamond p \pause\). Presumably not. This means that
Hero knows that \(p\) might be true, but she thinks \(p\) is false. So
false beliefs are still a problem.
\end{frame}

\begin{frame}{Transitivity}
\protect\hypertarget{transitivity}{}
Remember there is a tight connection between these two claims.

\begin{enumerate}
\tightlist
\item
  \(r\) is transitive, i.e., if \(wrx\) and \(xry\) then \(wry\).
\item
  It is always true that \(\Box A \supset \Box \Box A \pause\).
\end{enumerate}

Question:

\begin{itemize}
\tightlist
\item
  Should this hold for the epistemic interpretation of \(\Box\)?
\end{itemize}
\end{frame}

\begin{frame}{Current Debate}
\protect\hypertarget{current-debate}{}
\begin{itemize}
\tightlist
\item
  This is a much disputed question in the contemporary literature.
\item
  Here's one way to think about it.
\item
  Is there some state of affairs that (a) Hero knows does not obtain,
  but (b) would be possible if the world was some other way, and (c) for
  all Hero knows the world is that way? \pause
\item
  For a long time, philosophers (and computer scientists, economists
  etc) thought this was impossible, so they thought we should have
  \(\Box A \supset \Box \Box A \pause\). But recently a number of
  philosophers have started thinking this isn't true.
\end{itemize}
\end{frame}

\begin{frame}{Outside Philosophy}
\protect\hypertarget{outside-philosophy}{}
This is a philosophy course, so I'm going to focus on what happens in
philosophy.

\begin{itemize}
\tightlist
\item
  But the issue has broader ramifications.
\item
  Lots of disciplines use models that include a representation of what
  various agents (living or artificial) know at different points.
\item
  And the standard way this is done doesn't even allow for the
  representation of transitivity failures.
\item
  Indeed, it doesn't even allow for symmetry failures, which is a bigger
  problem.
\item
  But the issue here is one that lots of theorists should worry about.
\end{itemize}
\end{frame}

\begin{frame}{Margins of Error}
\protect\hypertarget{margins-of-error}{}
Think about the following situation.

\begin{itemize}
\tightlist
\item
  Hero is in a large lecture - maybe intro philosophy.
\item
  As often happens, they are a bit bored, and start estimating how many
  people are in the lecture theatre.
\item
  They are good at this kind of estimation, but not perfect.
\item
  But they are almost always within 10\% of the correct count. That's
  their \textbf{margin of error}.
\item
  Today they guess there are 200 students in the lecture, and (a little
  surprisingly) there are indeed exactly 200 students in the lecture.
\end{itemize}
\end{frame}

\begin{frame}{Margins and Knowledge}
\protect\hypertarget{margins-and-knowledge}{}
What does Hero know?

\begin{itemize}
\tightlist
\item
  Presumably they don't know that there are precisely 200 students in
  the lecture.
\item
  After all, they aren't usually this accurate.
\item
  But they do know that the number of students in somewhere between 180
  and 220, since they are accurate to within 10\%.
\end{itemize}
\end{frame}

\begin{frame}{Some worlds}
\protect\hypertarget{some-worlds}{}
Let \(w_n\) be the world where there are precisely \(n\) students in the
lecture.

\begin{itemize}
\tightlist
\item
  So the actual world is \(w_{200}\).
\item
  And every world between \(w_{180}\) and \(w_{220}\) is possible.
  \pause 
\item
  But the worlds outside that range are not possible - Hero knows they
  do not obtain. \pause Now think about which worlds are possible from
  \(w_{220}\). \pause 
\item
  In particular, should we think \(w_{220}rw_{230}\)?
\end{itemize}
\end{frame}

\begin{frame}{What's At Stake}
\protect\hypertarget{whats-at-stake}{}
If \(w_{220}rw_{230}\) then transitivity fails.

\begin{itemize}
\tightlist
\item
  We have \(w_{200}rw_{220}\).
\item
  And we have, by hypothesis, \(w_{220}rw_{230}\).
\item
  But we don't have \(w_{200}rw_{230}\), since Hero actually knows that
  there are less than 230 students in the lecture. \pause 
\end{itemize}

But if we don't have \(w_{220}rw_{230}\), then transitivity holds here.
And this kind of case might be the best case for a transitivity failure.
\end{frame}

\begin{frame}{So What's Right?}
\protect\hypertarget{so-whats-right}{}
Think about what Hero knows in the world where

\begin{enumerate}
\tightlist
\item
  There are actually 220 students in the lecture, and
\item
  She guesses there are 200 students. \pause 
\end{enumerate}

\begin{itemize}
\tightlist
\item
  On the one hand, her estimations are only accurate to within 10\%, so
  it would be funny if she knew there were less than 230 students. That
  would mean she could rule out something that was within the margin of
  error of her estimation.
\end{itemize}
\end{frame}

\begin{frame}{So What's Right?}
\protect\hypertarget{so-whats-right-1}{}
Think about what Hero knows in the world where

\begin{enumerate}
\tightlist
\item
  There are actually 220 students in the lecture, and
\item
  She guesses there are 200 students. \pause 
\end{enumerate}

\begin{itemize}
\tightlist
\item
  On the other hand, she knows that she guessed 200. And she knows (we
  can assume) that she's almost always within 10\% of the truth. While
  she does not know this is one of the cases where she is this accurate,
  it actually is such a case. So maybe she can knowingly deduce (from
  her guess) that the crowd size is between 180 and 220.
\end{itemize}
\end{frame}

\begin{frame}{Other Applications}
\protect\hypertarget{other-applications}{}
This might seem like a weird special case, but it comes up literally all
the time.

\begin{itemize}
\tightlist
\item
  Consider any case where Hero uses a measuring device (scales to weigh
  something, a gauge in their car to measure speed, or anything of the
  sort) that has a known margin of error \(m\).
\item
  And consider what happens when the real value is \(a\), and the
  displayed value is \(b\), with \(|a - b| < m\).
\end{itemize}
\end{frame}

\begin{frame}{What does Hero know?}
\protect\hypertarget{what-does-hero-know}{}
\begin{itemize}
\tightlist
\item
  One option: Hero knows that real value is in \([b-m, b+m]\). The
  machine is working, and when it does, the user knows that the real
  value is within the margin of error of the diplayed value.
\item
  Another option: Hero knows that the real value is in
  \([b-m-e, b+m+e]\), where \(e = |a - b|\) is the error of the machine
  on this occasion. This way, Hero can never rule out anything closer to
  reality than the margin of error of the machine.
\item
  And there are other options.
\end{itemize}
\end{frame}

\begin{frame}{Implications for Epistemic Logic}
\protect\hypertarget{implications-for-epistemic-logic}{}
\begin{itemize}
\tightlist
\item
  Option 1 strongly supports \(r\) being transitive; indeed if we can
  (somehow) rule out false beliefs, it even suggests that \(r\) is an
  equivalence relation.
\item
  Option 2 means that \(r\) will not be transitive.
\item
  And the case is hardly an obscure one; most measuring devices have a
  non-zero margin of error and on an occasion are not accurate to
  arbitrarily many decimal places.
\end{itemize}
\end{frame}

\begin{frame}{A Verdict?}
\protect\hypertarget{a-verdict}{}
I'm not going to adjudicate this here. Indeed it's an open debate in the
literature.

\begin{itemize}
\tightlist
\item
  What I do think is that thinking about these models, and in particular
  how they can be used to model noisy estimation of point values, is a
  helpful way to approach the problem.
\item
  It's more helpful than thinking directly about whether someone could
  know something without knowing that they knew it.
\end{itemize}
\end{frame}



\end{document}
