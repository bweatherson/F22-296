% Options for packages loaded elsewhere
\PassOptionsToPackage{unicode}{hyperref}
\PassOptionsToPackage{hyphens}{url}
%
\documentclass[
  14pt,
  letterpaper,
  ignorenonframetext,
  aspectratio=169,
  handout]{beamer}
\usepackage{pgfpages}
\setbeamertemplate{caption}[numbered]
\setbeamertemplate{caption label separator}{: }
\setbeamercolor{caption name}{fg=normal text.fg}
\beamertemplatenavigationsymbolsempty
% Prevent slide breaks in the middle of a paragraph
\widowpenalties 1 10000
\raggedbottom
\setbeamertemplate{part page}{
  \centering
  \begin{beamercolorbox}[sep=16pt,center]{part title}
    \usebeamerfont{part title}\insertpart\par
  \end{beamercolorbox}
}
\setbeamertemplate{section page}{
  \centering
  \begin{beamercolorbox}[sep=12pt,center]{part title}
    \usebeamerfont{section title}\insertsection\par
  \end{beamercolorbox}
}
\setbeamertemplate{subsection page}{
  \centering
  \begin{beamercolorbox}[sep=8pt,center]{part title}
    \usebeamerfont{subsection title}\insertsubsection\par
  \end{beamercolorbox}
}
\AtBeginPart{
  \frame{\partpage}
}
\AtBeginSection{
  \ifbibliography
  \else
    \frame{\sectionpage}
  \fi
}
\AtBeginSubsection{
  \frame{\subsectionpage}
}

\usepackage{amsmath,amssymb}
\usepackage{lmodern}
\usepackage{iftex}
\ifPDFTeX
  \usepackage[T1]{fontenc}
  \usepackage[utf8]{inputenc}
  \usepackage{textcomp} % provide euro and other symbols
\else % if luatex or xetex
  \usepackage{unicode-math}
  \defaultfontfeatures{Scale=MatchLowercase}
  \defaultfontfeatures[\rmfamily]{Ligatures=TeX,Scale=1}
  \setmainfont[Scale=MatchLowercase]{SF Pro Text Light}
  \setmathfont[]{STIX Two Math}
\fi
\usetheme[]{boxes}
\usecolortheme{wolverine}
\usefonttheme{serif} % use mainfont rather than sansfont for slide text
% Use upquote if available, for straight quotes in verbatim environments
\IfFileExists{upquote.sty}{\usepackage{upquote}}{}
\IfFileExists{microtype.sty}{% use microtype if available
  \usepackage[]{microtype}
  \UseMicrotypeSet[protrusion]{basicmath} % disable protrusion for tt fonts
}{}
\makeatletter
\@ifundefined{KOMAClassName}{% if non-KOMA class
  \IfFileExists{parskip.sty}{%
    \usepackage{parskip}
  }{% else
    \setlength{\parindent}{0pt}
    \setlength{\parskip}{6pt plus 2pt minus 1pt}}
}{% if KOMA class
  \KOMAoptions{parskip=half}}
\makeatother
\usepackage{xcolor}
\newif\ifbibliography
\setlength{\emergencystretch}{3em} % prevent overfull lines
\setcounter{secnumdepth}{-\maxdimen} % remove section numbering


\providecommand{\tightlist}{%
  \setlength{\itemsep}{0pt}\setlength{\parskip}{0pt}}\usepackage{longtable,booktabs,array}
\usepackage{calc} % for calculating minipage widths
\usepackage{caption}
% Make caption package work with longtable
\makeatletter
\def\fnum@table{\tablename~\thetable}
\makeatother
\usepackage{graphicx}
\makeatletter
\def\maxwidth{\ifdim\Gin@nat@width>\linewidth\linewidth\else\Gin@nat@width\fi}
\def\maxheight{\ifdim\Gin@nat@height>\textheight\textheight\else\Gin@nat@height\fi}
\makeatother
% Scale images if necessary, so that they will not overflow the page
% margins by default, and it is still possible to overwrite the defaults
% using explicit options in \includegraphics[width, height, ...]{}
\setkeys{Gin}{width=\maxwidth,height=\maxheight,keepaspectratio}
% Set default figure placement to htbp
\makeatletter
\def\fps@figure{htbp}
\makeatother

\captionsetup[figure]{labelformat=empty}
\usepackage{pgfpages}
\setbeamertemplate{itemize item}[circle]
\setbeamertemplate{footline}[frame number]{}
\mode<handout>{\pgfpagesuselayout{6 on 1}[letterpaper, border shrink=8mm]}
\AtBeginSection{%
   \begin{frame}
       \tableofcontents[currentsection]
   \end{frame}
}
\makeatletter
\makeatother
\makeatletter
\makeatother
\makeatletter
\@ifpackageloaded{caption}{}{\usepackage{caption}}
\AtBeginDocument{%
\ifdefined\contentsname
  \renewcommand*\contentsname{Table of contents}
\else
  \newcommand\contentsname{Table of contents}
\fi
\ifdefined\listfigurename
  \renewcommand*\listfigurename{List of Figures}
\else
  \newcommand\listfigurename{List of Figures}
\fi
\ifdefined\listtablename
  \renewcommand*\listtablename{List of Tables}
\else
  \newcommand\listtablename{List of Tables}
\fi
\ifdefined\figurename
  \renewcommand*\figurename{Figure}
\else
  \newcommand\figurename{Figure}
\fi
\ifdefined\tablename
  \renewcommand*\tablename{Table}
\else
  \newcommand\tablename{Table}
\fi
}
\@ifpackageloaded{float}{}{\usepackage{float}}
\floatstyle{ruled}
\@ifundefined{c@chapter}{\newfloat{codelisting}{h}{lop}}{\newfloat{codelisting}{h}{lop}[chapter]}
\floatname{codelisting}{Listing}
\newcommand*\listoflistings{\listof{codelisting}{List of Listings}}
\makeatother
\makeatletter
\@ifpackageloaded{caption}{}{\usepackage{caption}}
\@ifpackageloaded{subcaption}{}{\usepackage{subcaption}}
\makeatother
\makeatletter
\@ifpackageloaded{tcolorbox}{}{\usepackage[many]{tcolorbox}}
\makeatother
\makeatletter
\@ifundefined{shadecolor}{\definecolor{shadecolor}{rgb}{.97, .97, .97}}
\makeatother
\makeatletter
\makeatother
\ifLuaTeX
  \usepackage{selnolig}  % disable illegal ligatures
\fi
\IfFileExists{bookmark.sty}{\usepackage{bookmark}}{\usepackage{hyperref}}
\IfFileExists{xurl.sty}{\usepackage{xurl}}{} % add URL line breaks if available
\urlstyle{same} % disable monospaced font for URLs
\hypersetup{
  pdftitle={Honors Logic, Lecture 09 - Modal Logic},
  pdfauthor={Brian Weatherson},
  hidelinks,
  pdfcreator={LaTeX via pandoc}}

\title{Honors Logic, Lecture 09 - Modal Logic}
\author{Brian Weatherson}
\date{2022-09-28}

\begin{document}
\frame{\titlepage}
\ifdefined\Shaded\renewenvironment{Shaded}{\begin{tcolorbox}[interior hidden, borderline west={3pt}{0pt}{shadecolor}, enhanced, boxrule=0pt, sharp corners, breakable, frame hidden]}{\end{tcolorbox}}\fi

\begin{frame}{What Modal Logic Is}
\protect\hypertarget{what-modal-logic-is}{}
The logics of \textbf{must} and \textbf{might}. \pause

\begin{itemize}[<+->]
\tightlist
\item
  Why plural? Because we do not assume that these words have a single
  determinate meaning.
\end{itemize}
\end{frame}

\begin{frame}{Examples of Must}
\protect\hypertarget{examples-of-must}{}
\begin{enumerate}[<+->]
\tightlist
\item
  If \(x = 2 + 2\), then \(x\) must equal 4.
\item
  If something is a cat, then it must be a mammal.
\item
  If the gardener is innocent, then it must be the butler who did it.
\item
  You must drive under 70mph on I-94.
\item
  You must keep your promises.
\item
  If you set out a knife and fork, the fork must go on the left. \pause
\end{enumerate}

To my ears, 1 is \textbf{logical} necessity, 2 is \textbf{metaphysical}
necessity, 3 is \textbf{epistemic} necessity, 4 is \textbf{legal}
necessity, 5 is \textbf{moral} (or \textbf{deontic}) necessity and 6 is
\textbf{etiquette} necessity.
\end{frame}

\begin{frame}{Examples of May/Might}
\protect\hypertarget{examples-of-maymight}{}
\begin{enumerate}[<+->]
\tightlist
\item
  If \(x\) is prime, then \(x\) might be even.
\item
  If \(x\) is a cat, then \(x\) might be male.
\item
  It might be the butler or the gardener that did it.
\item
  You may drive at any speed below 30mph on State Street.
\item
  You may lie to save a friend's life.
\item
  You may use white napkins or red napkins.\pause
\end{enumerate}

To my ears, 1 is \textbf{logical} possibility, 2 is
\textbf{metaphysical} possibility, 3 is \textbf{epistemic} possibility,
4 is \textbf{legal} possibility, 5 is \textbf{moral} (or
\textbf{deontic}) possibility and 6 is \textbf{etiquette} possibility
(though I'm not sure about any of these).
\end{frame}

\begin{frame}{Logics}
\protect\hypertarget{logics}{}
Consider this very general claim.

\begin{quote}
If something must be true, then it is true.\pause
\end{quote}

\begin{itemize}[<+->]
\tightlist
\item
  That's true on the logical, epistemic and metaphysical interpretations
  of modality. \pause Indeed, it's something like a logical truth of
  those domains.\pause
\item
  But it is very much not true on the legal, moral or etiquette
  interpretations.\pause
\end{itemize}

So we want some logics where it is a logical truth, and some where it is
not.
\end{frame}

\begin{frame}{Language}
\protect\hypertarget{language}{}
We extend our language with two new operators: \(\Box\) and
\(\Diamond\).

\begin{itemize}[<+->]
\tightlist
\item
  If \(p\) is a sentence, so is \(\Box p\) and so is \(\Diamond p\).
\item
  These mean, respectively, that \(p\) must be true, and that \(p\)
  might be true.
\item
  We interpret these somewhat similar to negations; they just bind what
  they are immediately next to.
\item
  So \(\Box p \rightarrow q\) means \((\Box p) \rightarrow q\), not
  \(\Box(p \rightarrow q)\).
\end{itemize}
\end{frame}

\begin{frame}{Truth}
\protect\hypertarget{truth}{}
What does it take for these sentences to be true?
\end{frame}

\begin{frame}{Worlds}
\protect\hypertarget{worlds}{}
We start with Leibniz's idea that necessity is truth in all possible
worlds.

\begin{itemize}[<+->]
\tightlist
\item
  Leibniz was interested in metaphysical necessity, so we'll have to
  qualify this a little, but it's a good idea.
\item
  So instead of saying that each proposition simply has a truth value,
  we'll say that there are many \textbf{worlds}, and at each world each
  proposition has a truth value.
\item
  But don't assume that propositions have the same truth value at each
  world.
\item
  In one world I might be standing, and in another world I might be
  sitting.
\end{itemize}
\end{frame}

\begin{frame}{What Are Worlds}
\protect\hypertarget{what-are-worlds}{}
We are well and truly not going to get into the metaphysics of worlds
here.

\begin{itemize}[<+->]
\tightlist
\item
  Indeed, they need not even be anything like possible worlds in the
  sense that metaphysicians usually care about.
\item
  They might, for instance, be different times.
\item
  All we care about is that they are things at which propositions can be
  true or false.
\end{itemize}
\end{frame}

\begin{frame}{Valuations}
\protect\hypertarget{valuations}{}
A valuation function tells us which worlds atomic sentences are true at.

\begin{itemize}[<+->]
\tightlist
\item
  These can be completely arbitrary; we don't put any restrictions on
  them.
\end{itemize}
\end{frame}

\begin{frame}{Truth at a World}
\protect\hypertarget{truth-at-a-world}{}
We want more generally a function that tells us whether a sentence is
true at a particular world.

\begin{itemize}[<+->]
\tightlist
\item
  For sentences built up using \(\wedge, \vee, \rightarrow, \neg\), this
  is relatively easy.
\item
  We just keep on using truth tables.
\item
  So if at world \(w\), \(A\) is true and \(B\) is false, then
  \(A \wedge B\) is false and \(A \vee B\) is true.
\end{itemize}
\end{frame}

\begin{frame}{Modal Values}
\protect\hypertarget{modal-values}{}
We also need values for these sentences:

\begin{itemize}[<+->]
\tightlist
\item
  \(\Box A\)
\item
  \(\Diamond A\)
\end{itemize}

It turns out these are more complicated - but not much more complicated.
\end{frame}

\begin{frame}{Accessibility}
\protect\hypertarget{accessibility}{}
The last part of our model is an \textbf{accessibility} relation between
worlds.

\begin{itemize}[<+->]
\tightlist
\item
  Again, this can be completely arbitrary.
\item
  We don't yet put any restrictions on it.
\item
  Notably, we don't assume that it is \textbf{reflexive},
  \textbf{symmetric} or \textbf{transitive}
\end{itemize}
\end{frame}

\begin{frame}{Properties of Relations}
\protect\hypertarget{properties-of-relations}{}
\begin{itemize}[<+->]
\tightlist
\item
  \(R\) is reflexive iff for all \(x\), \(xRx\).\pause
\item
  \(R\) is symmetric iff for all \(x, y\), if \(xRy\) then
  \(yRx\).\pause
\item
  \(R\) is transitive iff for all \(x, y, z\) if \(xRy\) and \(yRz\)
  then \(xRz\).\pause
\end{itemize}

A lot of relations we care about have one or more of these properties,
but not all do. Consider, for example, \textbf{admires} as an example of
a relation with none of them.
\end{frame}

\begin{frame}{Truth of Modal Formulas}
\protect\hypertarget{truth-of-modal-formulas}{}
A sentence \(\Box A\) is true at a world \(x\) just in case the
following condition is met:

\begin{itemize}[<+->]
\tightlist
\item
  For all worlds \(y\) such that \(xRy\), \(A\) is true at world
  \(y\).\pause
\end{itemize}

A sentence \(\Diamond A\) is true at a world \(x\) just in case the
following condition is met:

\begin{itemize}[<+->]
\tightlist
\item
  For some world \(y\) such that \(xRy\), \(A\) is true at world \(y\).
\end{itemize}
\end{frame}

\begin{frame}{Modal Truth}
\protect\hypertarget{modal-truth}{}
\begin{itemize}[<+->]
\tightlist
\item
  Something is necessarily true iff it is true everywhere that is
  accessible.
\item
  Something is possibly true iff it is true somewhere accessible. \pause
\end{itemize}

We get back the Leibnizian idea that necessity is truth in all possible
worlds if we assume the accessibility relation is the universal
relation, i.e., \(xRy\) for all \(x, y\).
\end{frame}

\begin{frame}{Metaphysical Necessity}
\protect\hypertarget{metaphysical-necessity}{}
On this Leibnizian model, where all worlds can access all worlds,
iterated modalities are rather uninteresting. These three sentences are
true in the same worlds/models.

\begin{enumerate}[<+->]
\tightlist
\item
  \(\Box A\)
\item
  \(\Box \Box A\)
\item
  \(\Diamond \Box A\)
\end{enumerate}

\begin{itemize}[<+->]
\tightlist
\item
  That's because if \(\Box A\) is true at any world, then it is true at
  all worlds. In the general case, where we do not assume that \(R\) is
  universal, these are not equivalent.
\end{itemize}
\end{frame}

\begin{frame}{For Next Time}
\protect\hypertarget{for-next-time}{}
We'll talk about the relationship between boxes and diamonds.
\end{frame}



\end{document}
